\documentclass[a4paper]{article}
\usepackage{import}
\usepackage{graphicx}
\usepackage{float}
\usepackage{pgfplots}
\usepackage{listings}
\usepackage{enumitem}
\usepackage{tikz}
\usetikzlibrary{decorations.pathreplacing} % for angle arc
\usetikzlibrary{angles, quotes, calc, positioning, trees} % for drawing angles
\pgfplotsset{compat=1.18,width=10cm}
\usepackage{tikz-cd}
\usepackage{booktabs}
\usepackage{cancel}
\usepackage{amsmath}
\usepackage{csquotes}
\usepackage{gensymb}
\usepackage{forest}
\usepackage{amsthm}
\usepackage{amssymb}
\usepackage{pgfplots}
\usepackage{lipsum}
\usepackage{mdframed} 
\usepackage{color}   
\usepackage{hyperref}
\newmdtheoremenv{theo}{Theorem}
\usepackage{mathtools}
\DeclarePairedDelimiter\ceil{\lceil}{\rceil}
\DeclarePairedDelimiter\floor{\lfloor}{\rfloor}

\hypersetup{
    colorlinks=true, %set true if you want colored links
    linktoc=all,     %set to all if you want both sections and subsections linked
    linkcolor=black,  %choose some color if you want links to stand out
}

% Define theorem styles
\newtheorem{theorem}{Theorem}[section]    % Theorems numbered within sections
\newtheorem{lemma}[theorem]{Lemma}        % Lemmas use the same counter as theorems
\newtheorem{corollary}[theorem]{Corollary} % Corollaries use the same counter as theorems
\newtheorem{proposition}[theorem]{Proposition} % Proposition uses the same counter
\newtheorem{property}[theorem]{Property}
\theoremstyle{definition}
\newtheorem{definition}[theorem]{Definition} % Now uses the same counter as theorems


% Remark-style theorem
\theoremstyle{remark}
\newtheorem{remark}[theorem]{Remark}

% Boxed environment for theorems
\newmdenv[
  linewidth=0.8pt,
  roundcorner=5pt,
  linecolor=black,
  backgroundcolor=white!5,
  skipabove=\baselineskip,
  skipbelow=\baselineskip,
  innerleftmargin=10pt,
  innerrightmargin=10pt,
  innertopmargin=5pt,
  innerbottommargin=5pt
]{thmbox}

% Custom proof environment (also boxed)
\renewenvironment{proof}[1][Proof]{%
  \begin{mdframed}[linewidth=0.8pt, roundcorner=5pt, linecolor=black, skipabove=\baselineskip, skipbelow=\baselineskip, innertopmargin=5pt, innerbottommargin=5pt]%
  \noindent\textbf{#1. }%
}{%
  \end{mdframed}%
}

% Redefine theorem environments to use thmbox
\let\oldtheorem\theorem
\renewenvironment{theorem}{\begin{thmbox}\begin{oldtheorem}}{\end{oldtheorem}\end{thmbox}}

\let\oldlemma\lemma
\renewenvironment{lemma}{\begin{thmbox}\begin{oldlemma}}{\end{oldlemma}\end{thmbox}}

\let\oldcorollary\corollary
\renewenvironment{corollary}{\begin{thmbox}\begin{oldcorollary}}{\end{oldcorollary}\end{thmbox}}

\let\oldproposition\proposition
\renewenvironment{proposition}{\begin{thmbox}\begin{oldproposition}}{\end{oldproposition}\end{thmbox}}

\let\oldproperty\property
  \renewenvironment{property}{\begin{oldproperty}}{\end{oldproperty}}


% Reference shortcuts
\newcommand{\thmref}[1]{Theorem~\ref{#1}}
\newcommand{\lemref}[1]{Lemma~\ref{#1}}
\newcommand{\corref}[1]{Corollary~\ref{#1}}
\newcommand{\propref}[1]{Property~\ref{#1}} 

% To customize QED symbol
\renewcommand{\qedsymbol}{$\blacksquare$}

\usetikzlibrary{decorations.pathreplacing} % for angle arc
\usetikzlibrary{angles, quotes, calc} % for drawing angles

\usepackage{color}   %May be necessary if you want to color links
\usepackage{hyperref}
\hypersetup{
    colorlinks=true, %set true if you want colored links
    linktoc=all,     %set to all if you want both sections and subsections linked
    linkcolor=black,  %choose some color if you want links to stand out
}

\usepackage{xcolor}
\usepackage[most]{tcolorbox}
% Define a custom tcolorbox environment for examples
\newtcolorbox{examplebox}[2][]{
  colback=blue!5!white,
  colframe=blue!30!black,
  title=#2,
  boxrule=0mm,
  fonttitle=\bfseries,
  width=\textwidth,
  breakable,
  #1
}

\newtcolorbox{definizione}[2] {
  colback=green!5!white,
  colframe=green!30!black,
  title=#2,
  boxrule=0mm,
  fonttitle=\bfseries,
  width=\textwidth,
  breakable,
  #1
}

\definecolor{codegreen}{rgb}{0,0.6,0}
\definecolor{codegray}{rgb}{0.5,0.5,0.5}
\definecolor{codepurple}{rgb}{0.58,0,0.82}
\definecolor{backcolour}{rgb}{0.95,0.95,0.92}

\lstdefinestyle{mystyle}{
    backgroundcolor=\color{backcolour},   
    commentstyle=\color{codegreen},
    keywordstyle=\color{magenta},
    numberstyle=\tiny\color{codegray},
    stringstyle=\color{codepurple},
    basicstyle=\ttfamily\footnotesize,
    breakatwhitespace=false,         
    breaklines=true,                 
    captionpos=b,                    
    keepspaces=true,                 
    numbers=left,                    
    numbersep=5pt,                  
    showspaces=false,                
    showstringspaces=false,
    showtabs=false,                  
    tabsize=2
}

\lstset{style=mystyle}

\makeatletter
\renewcommand*\env@matrix[1][*\c@MaxMatrixCols c]{%
  \hskip -\arraycolsep
  \let\@ifnextchar\new@ifnextchar
  \array{#1}}
\makeatother


\onehalfspacing
\title{Fondamenti di Informatica}
\author{Università di Verona\\Imbriani Paolo - VR500437\\Professor Isabella Mastroeni}

\begin{document}

\begin{figure}
    \centering
    \includegraphics[width=0.3\textwidth]{../UniversityofVerona.png}
    \label{fig:centered-image}
\end{figure}

\maketitle

\pagebreak

\tableofcontents

\pagebreak

\section{Cosa è l'informatica?}

La domanda chiave di questo corso è: \textit{``Cosa è l'informatica?''}. 
Ci sono diversi definizioni a seconda del contesto, ma in generale l'informatica è lo studio dei processi che trasformano l'informazione.
Possiamo vedere, storicamente, diverse definizioni come quella in inglese come "Computer Science" che vede
l'informatica come studio della calcolabilità, della computazione e dell'informazione.

\subsection{Perché la calcolabilità}

Si studia la calcolabilità perché ci aiuta a capire cosa possiamo fare con gli strumenti che abbiamo.
Quando descriviamo un programma, quanto tempo ci mette e quanto spazio utilizza è una domanda importante 
per capire se il programma è efficiente o meno. Anche in senso dei linguaggi di programmazione per capire se usiamo
quello giusto per il problema che stiamo cercando di risolvere. Chiaramente è un qualcosa che in continuo
sviluppo perché si evolve in base alla tecnologia che abbiamo a disposizione.
\\
\\
Uno dei pionieri è stato \textbf{Hilbert} che si chiedeva se la matematica fosse formalizzabile come insieme
finito (non contraddittorio) di assiomi? Godel dimostrò che non è possibile rappresentare la matematica 
come un insieme finito di assiomi in maniera non contradditoria, dicendo che in ogni sistema formale
ci sono proposizioni vere che non possono essere dimostrate all'interno del sistema.
\\
\\
Nel tentativo di rispondere a queste (ed altre) domande si è costruito un modello che 
permette di comprendere profondamente ilr agionamento computazionale,
permettendo di applicarlo ad ogni disciplina.
Cosa è calcolabile e cosa non lo è? 

\subsubsection{Macchina di Turing} 

Anche Turing si pose questa domanda e propose la \textbf{Macchina di Turing} come modello di calcolo.
Una sola macchina (programmabile) per tutti i problemi.
La macchina è universale (interprete):
\[Init(P,x)\ = \begin{cases}
    P(x) & \text{se } P \text{ è un programma che termina}\\
    \uparrow & \text{se } P \text{ è un programma che non termina}
\end{cases}\]
Se un problema è intuitivamente calcolabile, allora esisterà una macchina di Turing (o un dispostivo
equivalente, come il computer) in grado di risolverlo, cioè calcolarlo.
I modelli equivalenti possono essere:
\begin{itemize}
    \item Lambda calcolo 
    \item Funzioni ricorsive
    \item Linguaggi di programmazione (Turing completi)
\end{itemize}
I problemi non calcolabili sono infinitamente più numerosi di quelli calcolabili.
\subsubsection{Limiti dell'informatica}

L'informatica ha più limiti di quanto si possa pensare, definiti dall'equivalenza di Turing
e l'incompletezza di Godel che ci dicono che non possiamo risolvere tutti i problemi.
Ci sono anche limiti fisici e tecnologici come:
\begin{itemize}
    \item Dati non osservabili (teorema di Shannon)
    \item Dati non controllabili (velocità della luce)
\end{itemize}


\end{document}