\documentclass[a4paper]{article}
\usepackage{import}
\usepackage{graphicx}
\usepackage{float}
\usepackage{pgfplots}
\usepackage{listings}
\usepackage{enumitem}
\usepackage{tikz}
\usetikzlibrary{decorations.pathreplacing} % for angle arc
\usetikzlibrary{angles, quotes, calc, positioning, trees} % for drawing angles
\pgfplotsset{compat=1.18,width=10cm}
\usepackage{tikz-cd}
\usepackage{booktabs}
\usepackage{cancel}
\usepackage{amsmath}
\usepackage{csquotes}
\usepackage{gensymb}
\usepackage{forest}
\usepackage{amsthm}
\usepackage{amssymb}
\usepackage{pgfplots}
\usepackage{lipsum}
\usepackage{mdframed} 
\usepackage{color}   
\usepackage{hyperref}
\newmdtheoremenv{theo}{Theorem}
\usepackage{mathtools}
\DeclarePairedDelimiter\ceil{\lceil}{\rceil}
\DeclarePairedDelimiter\floor{\lfloor}{\rfloor}

\hypersetup{
    colorlinks=true, %set true if you want colored links
    linktoc=all,     %set to all if you want both sections and subsections linked
    linkcolor=black,  %choose some color if you want links to stand out
}

% Define theorem styles
\newtheorem{theorem}{Theorem}[section]    % Theorems numbered within sections
\newtheorem{lemma}[theorem]{Lemma}        % Lemmas use the same counter as theorems
\newtheorem{corollary}[theorem]{Corollary} % Corollaries use the same counter as theorems
\newtheorem{proposition}[theorem]{Proposition} % Proposition uses the same counter
\newtheorem{property}[theorem]{Property}
\theoremstyle{definition}
\newtheorem{definition}[theorem]{Definition} % Now uses the same counter as theorems


% Remark-style theorem
\theoremstyle{remark}
\newtheorem{remark}[theorem]{Remark}

% Boxed environment for theorems
\newmdenv[
  linewidth=0.8pt,
  roundcorner=5pt,
  linecolor=black,
  backgroundcolor=white!5,
  skipabove=\baselineskip,
  skipbelow=\baselineskip,
  innerleftmargin=10pt,
  innerrightmargin=10pt,
  innertopmargin=5pt,
  innerbottommargin=5pt
]{thmbox}

% Custom proof environment (also boxed)
\renewenvironment{proof}[1][Proof]{%
  \begin{mdframed}[linewidth=0.8pt, roundcorner=5pt, linecolor=black, skipabove=\baselineskip, skipbelow=\baselineskip, innertopmargin=5pt, innerbottommargin=5pt]%
  \noindent\textbf{#1. }%
}{%
  \end{mdframed}%
}

% Redefine theorem environments to use thmbox
\let\oldtheorem\theorem
\renewenvironment{theorem}{\begin{thmbox}\begin{oldtheorem}}{\end{oldtheorem}\end{thmbox}}

\let\oldlemma\lemma
\renewenvironment{lemma}{\begin{thmbox}\begin{oldlemma}}{\end{oldlemma}\end{thmbox}}

\let\oldcorollary\corollary
\renewenvironment{corollary}{\begin{thmbox}\begin{oldcorollary}}{\end{oldcorollary}\end{thmbox}}

\let\oldproposition\proposition
\renewenvironment{proposition}{\begin{thmbox}\begin{oldproposition}}{\end{oldproposition}\end{thmbox}}

\let\oldproperty\property
  \renewenvironment{property}{\begin{oldproperty}}{\end{oldproperty}}


% Reference shortcuts
\newcommand{\thmref}[1]{Theorem~\ref{#1}}
\newcommand{\lemref}[1]{Lemma~\ref{#1}}
\newcommand{\corref}[1]{Corollary~\ref{#1}}
\newcommand{\propref}[1]{Property~\ref{#1}} 

% To customize QED symbol
\renewcommand{\qedsymbol}{$\blacksquare$}

\usetikzlibrary{decorations.pathreplacing} % for angle arc
\usetikzlibrary{angles, quotes, calc} % for drawing angles

\usepackage{color}   %May be necessary if you want to color links
\usepackage{hyperref}
\hypersetup{
    colorlinks=true, %set true if you want colored links
    linktoc=all,     %set to all if you want both sections and subsections linked
    linkcolor=black,  %choose some color if you want links to stand out
}

\usepackage{xcolor}
\usepackage[most]{tcolorbox}
% Define a custom tcolorbox environment for examples
\newtcolorbox{examplebox}[2][]{
  colback=blue!5!white,
  colframe=blue!30!black,
  title=#2,
  boxrule=0mm,
  fonttitle=\bfseries,
  width=\textwidth,
  breakable,
  #1
}

\newtcolorbox{definizione}[2] {
  colback=green!5!white,
  colframe=green!30!black,
  title=#2,
  boxrule=0mm,
  fonttitle=\bfseries,
  width=\textwidth,
  breakable,
  #1
}

\definecolor{codegreen}{rgb}{0,0.6,0}
\definecolor{codegray}{rgb}{0.5,0.5,0.5}
\definecolor{codepurple}{rgb}{0.58,0,0.82}
\definecolor{backcolour}{rgb}{0.95,0.95,0.92}

\lstdefinestyle{mystyle}{
    backgroundcolor=\color{backcolour},   
    commentstyle=\color{codegreen},
    keywordstyle=\color{magenta},
    numberstyle=\tiny\color{codegray},
    stringstyle=\color{codepurple},
    basicstyle=\ttfamily\footnotesize,
    breakatwhitespace=false,         
    breaklines=true,                 
    captionpos=b,                    
    keepspaces=true,                 
    numbers=left,                    
    numbersep=5pt,                  
    showspaces=false,                
    showstringspaces=false,
    showtabs=false,                  
    tabsize=2
}

\lstset{style=mystyle}

\makeatletter
\renewcommand*\env@matrix[1][*\c@MaxMatrixCols c]{%
  \hskip -\arraycolsep
  \let\@ifnextchar\new@ifnextchar
  \array{#1}}
\makeatother


\onehalfspacing
\title{Intelligenza Artificiale}
\author{Università di Verona\\Imbriani Paolo -VR500437\\Professor Alessandro Farinelli}

\begin{document}

\begin{figure}
    \centering
    \includegraphics[width=0.3\textwidth]{UniversityofVerona.png}
    \label{fig:centered-image}
\end{figure}

\maketitle

\pagebreak

\tableofcontents

\pagebreak

\section{Introduzione}

Alle origini dell'intelligenza artificiale vi è un bisogno diverso da quello che abbiamo oggi.
Alan Turing, negli anni 50 si era chiesto se le macchine potessero pensare, creando un test famoso ancora ora 
come "test di Turing" dove un interrogatore umano si deve interfacciare con un umano e una macchina 
e doveva capire chi dei due fosse chi. 
Nel 1956 ci fu uno studio fatto da il progetto di ricerca di Dartmouth, che aveva l'intento
di risolvere compiti che richiedeva l'intelligenza di una persona attraverso una macchina,
comprendendo che le \textit{anche le macchine possono imparare}.
La definizione più "accettata" di Intelligenza Artificiale è quella dove viene vista come una 
complessa e affascinante \textit{disciplina} che studia come simulare l'intelligeza in scenari complessi usando come
strumenti agenti autonomi per delle task ripetitive, sporche e pericolose che sfruttano l'analisi dei dati
(predizione e classificazione).

\dfn{}
{
    L'intelligenza artificiale è una disciplina che studia come \textbf{simulare} l'intelligenza
    umana in scenari complessi.
}
\noindent
Bisogna distinguere machine learning e programmazione:
\begin{itemize}
    \item \textbf{Programmazione}: macchine programmate per ogni task che devono eseguire (il concetto chiave
    è \textbf{il programma})
    \item \textbf{Machine Learning}: insegnare alla macchina (attraverso esempi) come risolvere task più complesse (il concetto chiave 
    è il \textbf{modello})
\end{itemize}

\subsection{Machine Learning}
L'idea di far apprendere una macchina si possono dividere in tre paradigmi contraddisti:
\begin{itemize}
    \item Unsupervised learning
    \item Supervised learning
    \item Reinforcement learning
\end{itemize}
Esistono poi i trasformatori, che sono modelli di machine learning probabilistici che si basano sul concetto di attenzione, che sono alla base di modelli come GPT.
Il concetto dell'attenzione è quello di dare più importanza ad alcune parole rispetto ad altre in un contesto, per esempio in una frase.
La potenza di questi trasformatori è che riescono a fare un'analisi del contesto molto più profonda rispetto ai modelli precedenti, 
permettendo di fare analisi di immagini come per esempio riconoscere oggetti in un'immagine o riconoscere dove è presente l'acqua
all'interno di una foto.

\subsection{Agenti intelligenti}

Un agente intelligente è un'entità che percepisce il suo ambiente attraverso dei sensori e agisce su di esso attraverso degli attuatori.
\begin{itemize}
    \item Percepisce l'ambiente attraverso dei \textbf{sensori}
    \item Agisce sull'ambiente attraverso degli \textbf{attuatori}
    \item Ha un \textbf{obiettivo} da raggiungere
\end{itemize}
Come dovrebbe comportarsi un agente intelligente?
\begin{itemize}
    \item \textbf{Razionale}: agisce per massimizzare il raggiungimento dell'obiettivo
    \item \textbf{Performance measure}: misura di quanto bene l'agente sta raggiungendo l'obiettivo
\end{itemize}
Quando vogliamo ragionare sul Reinforcement Learning, è utilire usare il \textit{Markov Decision Process}.
\dfn{}
{
    Un \textbf{Markov Decision Process (MDP)} è una tupla $(S, A, P, R)$ dove:
    \begin{itemize}
        \item $S$ è un insieme di stati
        \item $A$ è un insieme di azioni
        \item $P(s'|s,a)$ è la probabilità di transizione dallo stato $s$ allo stato $s'$ eseguendo l'azione $a$
        \item $R(s,a,s')$ è la ricompensa ottenuta eseguendo l'azione $a$ nello stato $s$ e transizionando nello stato $s'$
    \end{itemize}
}
\noindent
Poi si ha la \textit{policy} che è una funzione che mappa uno stato in un'azione. 







\end{document}