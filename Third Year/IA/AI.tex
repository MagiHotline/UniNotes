\documentclass[a4paper]{article}
\usepackage{import}
\usepackage{graphicx}
\usepackage{float}
\usepackage{pgfplots}
\usepackage{listings}
\usepackage{enumitem}
\usepackage{tikz}
\usetikzlibrary{decorations.pathreplacing} % for angle arc
\usetikzlibrary{angles, quotes, calc, positioning, trees} % for drawing angles
\pgfplotsset{compat=1.18,width=10cm}
\usepackage{tikz-cd}
\usepackage{booktabs}
\usepackage{cancel}
\usepackage{amsmath}
\usepackage{csquotes}
\usepackage{gensymb}
\usepackage{forest}
\usepackage{amsthm}
\usepackage{amssymb}
\usepackage{pgfplots}
\usepackage{lipsum}
\usepackage{mdframed} 
\usepackage{color}   
\usepackage{hyperref}
\newmdtheoremenv{theo}{Theorem}
\usepackage{mathtools}
\DeclarePairedDelimiter\ceil{\lceil}{\rceil}
\DeclarePairedDelimiter\floor{\lfloor}{\rfloor}

\hypersetup{
    colorlinks=true, %set true if you want colored links
    linktoc=all,     %set to all if you want both sections and subsections linked
    linkcolor=black,  %choose some color if you want links to stand out
}

% Define theorem styles
\newtheorem{theorem}{Theorem}[section]    % Theorems numbered within sections
\newtheorem{lemma}[theorem]{Lemma}        % Lemmas use the same counter as theorems
\newtheorem{corollary}[theorem]{Corollary} % Corollaries use the same counter as theorems
\newtheorem{proposition}[theorem]{Proposition} % Proposition uses the same counter
\newtheorem{property}[theorem]{Property}
\theoremstyle{definition}
\newtheorem{definition}[theorem]{Definition} % Now uses the same counter as theorems


% Remark-style theorem
\theoremstyle{remark}
\newtheorem{remark}[theorem]{Remark}

% Boxed environment for theorems
\newmdenv[
  linewidth=0.8pt,
  roundcorner=5pt,
  linecolor=black,
  backgroundcolor=white!5,
  skipabove=\baselineskip,
  skipbelow=\baselineskip,
  innerleftmargin=10pt,
  innerrightmargin=10pt,
  innertopmargin=5pt,
  innerbottommargin=5pt
]{thmbox}

% Custom proof environment (also boxed)
\renewenvironment{proof}[1][Proof]{%
  \begin{mdframed}[linewidth=0.8pt, roundcorner=5pt, linecolor=black, skipabove=\baselineskip, skipbelow=\baselineskip, innertopmargin=5pt, innerbottommargin=5pt]%
  \noindent\textbf{#1. }%
}{%
  \end{mdframed}%
}

% Redefine theorem environments to use thmbox
\let\oldtheorem\theorem
\renewenvironment{theorem}{\begin{thmbox}\begin{oldtheorem}}{\end{oldtheorem}\end{thmbox}}

\let\oldlemma\lemma
\renewenvironment{lemma}{\begin{thmbox}\begin{oldlemma}}{\end{oldlemma}\end{thmbox}}

\let\oldcorollary\corollary
\renewenvironment{corollary}{\begin{thmbox}\begin{oldcorollary}}{\end{oldcorollary}\end{thmbox}}

\let\oldproposition\proposition
\renewenvironment{proposition}{\begin{thmbox}\begin{oldproposition}}{\end{oldproposition}\end{thmbox}}

\let\oldproperty\property
  \renewenvironment{property}{\begin{oldproperty}}{\end{oldproperty}}


% Reference shortcuts
\newcommand{\thmref}[1]{Theorem~\ref{#1}}
\newcommand{\lemref}[1]{Lemma~\ref{#1}}
\newcommand{\corref}[1]{Corollary~\ref{#1}}
\newcommand{\propref}[1]{Property~\ref{#1}} 

% To customize QED symbol
\renewcommand{\qedsymbol}{$\blacksquare$}

\usetikzlibrary{decorations.pathreplacing} % for angle arc
\usetikzlibrary{angles, quotes, calc} % for drawing angles

\usepackage{color}   %May be necessary if you want to color links
\usepackage{hyperref}
\hypersetup{
    colorlinks=true, %set true if you want colored links
    linktoc=all,     %set to all if you want both sections and subsections linked
    linkcolor=black,  %choose some color if you want links to stand out
}

\usepackage{xcolor}
\usepackage[most]{tcolorbox}
% Define a custom tcolorbox environment for examples
\newtcolorbox{examplebox}[2][]{
  colback=blue!5!white,
  colframe=blue!30!black,
  title=#2,
  boxrule=0mm,
  fonttitle=\bfseries,
  width=\textwidth,
  breakable,
  #1
}

\newtcolorbox{definizione}[2] {
  colback=green!5!white,
  colframe=green!30!black,
  title=#2,
  boxrule=0mm,
  fonttitle=\bfseries,
  width=\textwidth,
  breakable,
  #1
}

\definecolor{codegreen}{rgb}{0,0.6,0}
\definecolor{codegray}{rgb}{0.5,0.5,0.5}
\definecolor{codepurple}{rgb}{0.58,0,0.82}
\definecolor{backcolour}{rgb}{0.95,0.95,0.92}

\lstdefinestyle{mystyle}{
    backgroundcolor=\color{backcolour},   
    commentstyle=\color{codegreen},
    keywordstyle=\color{magenta},
    numberstyle=\tiny\color{codegray},
    stringstyle=\color{codepurple},
    basicstyle=\ttfamily\footnotesize,
    breakatwhitespace=false,         
    breaklines=true,                 
    captionpos=b,                    
    keepspaces=true,                 
    numbers=left,                    
    numbersep=5pt,                  
    showspaces=false,                
    showstringspaces=false,
    showtabs=false,                  
    tabsize=2
}

\lstset{style=mystyle}

\makeatletter
\renewcommand*\env@matrix[1][*\c@MaxMatrixCols c]{%
  \hskip -\arraycolsep
  \let\@ifnextchar\new@ifnextchar
  \array{#1}}
\makeatother


\onehalfspacing
\title{Intelligenza Artificiale}
\author{Università di Verona\\Imbriani Paolo -VR500437\\Professor Alessandro Farinelli}

\begin{document}

\begin{figure}
    \centering
    \includegraphics[width=0.3\textwidth]{../UniversityofVerona.png}
    \label{fig:centered-image}
\end{figure}

\maketitle

\pagebreak

\tableofcontents

\pagebreak

\section{Introduzione}

Alle origini dell'intelligenza artificiale vi è un bisogno diverso da quello che abbiamo oggi.
Alan Turing, negli anni 50 si era chiesto se le macchine potessero pensare, creando un test famoso ancora ora 
come "test di Turing" dove un interrogatore umano si deve interfacciare con un umano e una macchina 
e doveva capire chi dei due fosse chi. 
Nel 1956 ci fu uno studio fatto da il progetto di ricerca di Dartmouth, che aveva l'intento
di risolvere compiti che richiedeva l'intelligenza di una persona attraverso una macchina,
comprendendo che le \textit{anche le macchine possono imparare}.
La definizione più "accettata" di Intelligenza Artificiale è quella dove viene vista come una 
complessa e affascinante \textit{disciplina} che studia come simulare l'intelligeza in scenari complessi usando come
strumenti agenti autonomi per delle task ripetitive, sporche e pericolose che sfruttano l'analisi dei dati
(predizione e classificazione).

\dfn{}
{
    L'intelligenza artificiale è una disciplina che studia come \textbf{simulare} l'intelligenza
    umana in scenari complessi.
}
\noindent
Bisogna distinguere machine learning e programmazione:
\begin{itemize}
    \item \textbf{Programmazione}: macchine programmate per ogni task che devono eseguire (il concetto chiave
    è \textbf{il programma})
    \item \textbf{Machine Learning}: insegnare alla macchina (attraverso esempi) come risolvere task più complesse (il concetto chiave 
    è il \textbf{modello})
\end{itemize}

\subsection{Machine Learning}
L'idea di far apprendere una macchina si possono dividere in tre paradigmi contraddisti:
\begin{itemize}
    \item Unsupervised learning
    \item Supervised learning
    \item Reinforcement learning
\end{itemize}
Esistono poi i trasformatori, che sono modelli di machine learning probabilistici che si basano sul concetto di attenzione, che sono alla base di modelli come GPT.
Il concetto dell'attenzione è quello di dare più importanza ad alcune parole rispetto ad altre in un contesto, per esempio in una frase.
La potenza di questi trasformatori è che riescono a fare un'analisi del contesto molto più profonda rispetto ai modelli precedenti, 
permettendo di fare analisi di immagini come per esempio riconoscere oggetti in un'immagine o riconoscere dove è presente l'acqua
all'interno di una foto.

\subsection{Agenti intelligenti}

Un agente intelligente è un'entità che percepisce il suo ambiente attraverso dei sensori e agisce su di esso attraverso degli attuatori.
\begin{itemize}
    \item Percepisce l'ambiente attraverso dei \textbf{sensori}
    \item Agisce sull'ambiente attraverso degli \textbf{attuatori}
    \item Ha un \textbf{obiettivo} da raggiungere
\end{itemize}
Come dovrebbe comportarsi un agente intelligente?
\begin{itemize}
    \item \textbf{Razionale}: agisce per massimizzare il raggiungimento dell'obiettivo
    \item \textbf{Performance measure}: misura di quanto bene l'agente sta raggiungendo l'obiettivo
\end{itemize}
Quando vogliamo ragionare sul Reinforcement Learning, è utilire usare il \textit{Markov Decision Process}.
\dfn{}
{
    Un \textbf{Markov Decision Process (MDP)} è una tupla $(S, A, P, R)$ dove:
    \begin{itemize}
        \item $S$ è un insieme di stati
        \item $A$ è un insieme di azioni
        \item $P(s'|s,a)$ è la probabilità di transizione dallo stato $s$ allo stato $s'$ eseguendo l'azione $a$
        \item $R(s,a,s')$ è la ricompensa ottenuta eseguendo l'azione $a$ nello stato $s$ e transizionando nello stato $s'$
    \end{itemize}
}
\noindent
Poi si ha la \textit{policy} che è una funzione che mappa uno stato in un'azione. 

\section{Risolvere problemi con la ricerca}

\subsection{Agents and enviroments}

Gli agenti includono umani, robot, softbot, termostati, ecc. La funzione
agente mappa la storia delle percezioni in azioni.
\[f : \mathcal{P}^* \mapsto A\]
Il \textit{programma dell'agente} viene eseguito su un'architettura fisica che produce
$f$.

\ex{}
{
    Immaginiamo di avere un agente aspirapolvere che percepisce
    il luogo e i suoi contenuti.
    \begin{itemize}
        \item \textbf{Percezioni}: bump, Dirty e location (A o B)
        \item \textbf{Azioni}: left, right, suck, noOp
    \end{itemize}
    un esempio di sequenza percepita potrebbe essere:
    \[(A, Dirty), Suck, (A, Dirty), Suck, (A, Clean), Right\] \[ 
    (B, Dirty), Suck, (B, Clean), Left, (A, Clean), NoOp\]
    Cosa fa la funzione \textit{Right}? Può essere implementata
    in un piccolo programma agente?
    Se un agentes ha $|\mathcal{P}|$ possibili percezioni, quante
    "entries" avrà la tabella della funzione agente dopo $T$ time steps?
    \[\sum_{t=1}^{T} |P|^t\]
    L'obiettivo dell'IA è quello di progettare \textbf{piccoli}
    programmi agenti che permettono di rappresentare grandi funzioni agenti.
}
\noindent
\begin{minted}{python}
function Reflex-Vacuum-Agent([location,status]) returns an action
    if status = dirty then return suck
    else if location = A then return right
    else if location = B then return left
\end{minted}
\noindent

\subsubsection{Multi-robot Patrolling}

\ex{}
{
    Considerate il seguente ambiente:
    \begin{itemize}
        \item Tre stanze (A,B,C) e due robot ($r_1, r_2$)
        \item $r_1$ può pattugliare $A$ e $B$, $r_2$ può pattugliare $B$ e $C$
        \item $r_1$ inizia da $A$ e $r_2$ inizia da $C$
        \item Il tempo di viaggio tra le stanze è $0$
        \item Performance Measure: minimizzare il tempo medio di inattività tra le stanze
        \item Media di inattività: somma degli intervalli nella quale la stanza non è stata visitata 
        da nessun robot
        \item Quale potrebbe essere un comportamento razionale di questo ambiente?
    \end{itemize}
    Quello che succede in maniera ragionevole è la seguente, dove S è la tupla
    in cui i robot sono posizionati:
    TODO
    \\
    Nei diversi casi si ha che il miglior modo per fare girare i robot
    è quello di farli muovere alternando chi entra nella stanza B 
    minimizzando anche la varianza nelle varie stanze perché dobbiamo
    stare attenti a non penalizzare troppo una stanza.
}

\subsection{Tipi di ambiente}

Il tipo di ambiente determina la progettazione di un agente? 
Nel mondo reale è ovviamente parzialmente visibile, stocastico, sequenziale,
dinamico, continuo, multi-agente.

\begin{itemize}
    \item \textbf{Completamente osservabile vs parzialmente osservabile}: un agente ha accesso
    completo allo stato dell'ambiente in ogni istante di tempo?
    \item \textbf{Deterministico vs stocastico}: il prossimo stato dell'ambiente è completamente
    determinato dallo stato corrente e dall'azione eseguita dall'agente?
    \item \textbf{Episodico vs sequenziale}: l'esperienza dell'agente è divisa in episodi
    indipendenti?
    \item \textbf{Statico vs dinamico}: l'ambiente può cambiare mentre l'agente sta pensando?
    \item \textbf{Discreto vs continuo}: il numero di stati, percezioni e azioni è finito o infinito?
    \item \textbf{Singolo agente vs multi-agente}: l'agente agisce da solo o ci sono altri agenti
    che possono influenzare l'ambiente?
\end{itemize}

\begin{table}[H]
    \centering
    \begin{tabular}{|c|c|c|c|c|}
        \hline
        & Crosswords & Robo-selector & Poker & Taxi \\
        \hline
        Osservabile & Sì  & Parziale & Parziale & Parziale \\
        Deterministico & Sì & No & No &  No\\
        Episodico & No & Sì & No & No\\
        Statico & Sì & No & Sì & No \\
        Discreto & Sì & No & Sì & No \\
        Singolo agente & Sì & Sì & No & No\\
        \hline
    \end{tabular}
\end{table}

\begin{itemize}
    \item Se il problema è deterministico e completamente osservabile, è un \textbf{single-state problem}
    \item Se il problema non è osservabile, è un \textbf{conformant problem}
    \item Se il problema è non deterministico o parzialmente osservabile, è un \textbf{contingency problem}
    \item Quando non conosco lo spazio degli stati è un \textbf{exploration problem}
\end{itemize}

\subsection{Problem Solving Agents}

Una forma ristretta di agente generale sono i: \textbf{Goal Based Agent}
\begin{itemize}
    \item Formula un goal e un problema partendo dallo stato corrente
    \item Cerco una soluzione a questo problema
    \item Eseguo la soluzione ignorando le percezioni
\end{itemize}
Notiamo che questo si chiama anche offline problem; la soluzione viene eseguita ad "occhi chiusi".
\begin{minted}{python}
function Simple-Problem-Solving-Agent(percept) returns an action
    static: solution, state, problem, action
    state <- Update-State(state, percept)
    if seq is empty then
        goal <- Formulate-Goal(state)
        problem <- Formulate-Problem(state, goal)
        seq <- Search(problem)
    action <- First(seq)
    seq <- Rest(seq)
    return action
\end{minted}

\ex{Vacanze in Romania}
{
    In viaggio in Romania, se attulamente ad Arad.
    Il viaggio parte domani da Bucharest.
    \begin{itemize}
        \item \textbf{Formulate Goal:} essere a Bucharest
        \item \textbf{Formulate Problem:} stati: varie città, azioni: guidare tra le città
        \item \textbf{Search:} trovare una sequenza di azioni che portano da Arad a Bucharest
        \item \textbf{Esempio di Soluzione:} Arad, Sibiu, Fagaras, Bucharest
    \end{itemize}
}

\subsubsection{Tree Search Algorithm}

Idea base: offline, esplorazione simulata di spazio di stati, generando
successori di stati già esplorati.
\begin{minted}{python}
function Tree-Search(problem) returns a solution, or failure
    initialize the frontier using the initial state of problem
    loop do
        if the frontier is empty then return failure
        node <- Pop an element from the frontier
        if problem.GOAL-TEST(node.STATE) then return SOLUTION(node)
        expand node, adding the resulting nodes to the frontier
    end
\end{minted}

\dfn{}
{
    Uno \textbf{stato} è una rappresentazione di una configurazione
    fisica.
}
\dfn{}
{
    Un \textbf{nodo} è una struttura dati che contiene:
    \begin{itemize}
        \item uno stato
        \item un puntatore al nodo genitore
        \item l'azione che ha generato lo stato
        \item il costo del cammino dal nodo radice a questo nodo
    \end{itemize}
    Gli stati non hanno parenti, azioni, figli, costi e profondità!
}

\begin{minted}{python}
function Expand(node, problem) returns a set of nodes
    successors <- an empty list
    for each action in problem.ACTIONS(node.STATE) do
        child <- CHILD-NODE(problem, node, action)
        add child to successors
    return successors    
\end{minted}
\noindent

\subsection{Strategie di ricerca}

Una strategia è definita dal scegliere l'ordine dei nodi di espansione.
Strategie vengono valutate insieme alle seguenti metriche:
\begin{itemize}
    \item Completezza: la strategia trova una soluzione se esiste
    \item Tempo: tempo di esecuzione della strategia
    \item Spazio: memoria usata dalla strategia
    \item Optimalità: la strategia trova la soluzione ottima?
\end{itemize}
Tempo e spazio sono misurati in termini di:
\begin{itemize}
    \item $b$ branching factor (numero massimo di figli per nodo)
    \item $d$ profondità della soluzione più superficiale
    \item $m$ profondità massima dell'albero di ricerca (potrebbe essere infinito)
\end{itemize}

\end{document}