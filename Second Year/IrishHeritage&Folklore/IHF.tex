\documentclass[a4paper]{article}
\usepackage{import}
\usepackage{graphicx}
\usepackage{float}
\usepackage{pgfplots}
\usepackage{listings}
\usepackage{enumitem}
\usepackage{tikz}
\usetikzlibrary{decorations.pathreplacing} % for angle arc
\usetikzlibrary{angles, quotes, calc, positioning, trees} % for drawing angles
\pgfplotsset{compat=1.18,width=10cm}
\usepackage{tikz-cd}
\usepackage{booktabs}
\usepackage{cancel}
\usepackage{amsmath}
\usepackage{csquotes}
\usepackage{gensymb}
\usepackage{forest}
\usepackage{amsthm}
\usepackage{amssymb}
\usepackage{pgfplots}
\usepackage{lipsum}
\usepackage{mdframed} 
\usepackage{color}   
\usepackage{hyperref}
\newmdtheoremenv{theo}{Theorem}
\usepackage{mathtools}
\DeclarePairedDelimiter\ceil{\lceil}{\rceil}
\DeclarePairedDelimiter\floor{\lfloor}{\rfloor}

\hypersetup{
    colorlinks=true, %set true if you want colored links
    linktoc=all,     %set to all if you want both sections and subsections linked
    linkcolor=black,  %choose some color if you want links to stand out
}

% Define theorem styles
\newtheorem{theorem}{Theorem}[section]    % Theorems numbered within sections
\newtheorem{lemma}[theorem]{Lemma}        % Lemmas use the same counter as theorems
\newtheorem{corollary}[theorem]{Corollary} % Corollaries use the same counter as theorems
\newtheorem{proposition}[theorem]{Proposition} % Proposition uses the same counter
\newtheorem{property}[theorem]{Property}
\theoremstyle{definition}
\newtheorem{definition}[theorem]{Definition} % Now uses the same counter as theorems


% Remark-style theorem
\theoremstyle{remark}
\newtheorem{remark}[theorem]{Remark}

% Boxed environment for theorems
\newmdenv[
  linewidth=0.8pt,
  roundcorner=5pt,
  linecolor=black,
  backgroundcolor=white!5,
  skipabove=\baselineskip,
  skipbelow=\baselineskip,
  innerleftmargin=10pt,
  innerrightmargin=10pt,
  innertopmargin=5pt,
  innerbottommargin=5pt
]{thmbox}

% Custom proof environment (also boxed)
\renewenvironment{proof}[1][Proof]{%
  \begin{mdframed}[linewidth=0.8pt, roundcorner=5pt, linecolor=black, skipabove=\baselineskip, skipbelow=\baselineskip, innertopmargin=5pt, innerbottommargin=5pt]%
  \noindent\textbf{#1. }%
}{%
  \end{mdframed}%
}

% Redefine theorem environments to use thmbox
\let\oldtheorem\theorem
\renewenvironment{theorem}{\begin{thmbox}\begin{oldtheorem}}{\end{oldtheorem}\end{thmbox}}

\let\oldlemma\lemma
\renewenvironment{lemma}{\begin{thmbox}\begin{oldlemma}}{\end{oldlemma}\end{thmbox}}

\let\oldcorollary\corollary
\renewenvironment{corollary}{\begin{thmbox}\begin{oldcorollary}}{\end{oldcorollary}\end{thmbox}}

\let\oldproposition\proposition
\renewenvironment{proposition}{\begin{thmbox}\begin{oldproposition}}{\end{oldproposition}\end{thmbox}}

\let\oldproperty\property
  \renewenvironment{property}{\begin{oldproperty}}{\end{oldproperty}}


% Reference shortcuts
\newcommand{\thmref}[1]{Theorem~\ref{#1}}
\newcommand{\lemref}[1]{Lemma~\ref{#1}}
\newcommand{\corref}[1]{Corollary~\ref{#1}}
\newcommand{\propref}[1]{Property~\ref{#1}} 

% To customize QED symbol
\renewcommand{\qedsymbol}{$\blacksquare$}

\usetikzlibrary{decorations.pathreplacing} % for angle arc
\usetikzlibrary{angles, quotes, calc} % for drawing angles

\usepackage{color}   %May be necessary if you want to color links
\usepackage{hyperref}
\hypersetup{
    colorlinks=true, %set true if you want colored links
    linktoc=all,     %set to all if you want both sections and subsections linked
    linkcolor=black,  %choose some color if you want links to stand out
}

\usepackage{xcolor}
\usepackage[most]{tcolorbox}
% Define a custom tcolorbox environment for examples
\newtcolorbox{examplebox}[2][]{
  colback=blue!5!white,
  colframe=blue!30!black,
  title=#2,
  boxrule=0mm,
  fonttitle=\bfseries,
  width=\textwidth,
  breakable,
  #1
}

\newtcolorbox{definizione}[2] {
  colback=green!5!white,
  colframe=green!30!black,
  title=#2,
  boxrule=0mm,
  fonttitle=\bfseries,
  width=\textwidth,
  breakable,
  #1
}

\definecolor{codegreen}{rgb}{0,0.6,0}
\definecolor{codegray}{rgb}{0.5,0.5,0.5}
\definecolor{codepurple}{rgb}{0.58,0,0.82}
\definecolor{backcolour}{rgb}{0.95,0.95,0.92}

\lstdefinestyle{mystyle}{
    backgroundcolor=\color{backcolour},   
    commentstyle=\color{codegreen},
    keywordstyle=\color{magenta},
    numberstyle=\tiny\color{codegray},
    stringstyle=\color{codepurple},
    basicstyle=\ttfamily\footnotesize,
    breakatwhitespace=false,         
    breaklines=true,                 
    captionpos=b,                    
    keepspaces=true,                 
    numbers=left,                    
    numbersep=5pt,                  
    showspaces=false,                
    showstringspaces=false,
    showtabs=false,                  
    tabsize=2
}

\lstset{style=mystyle}

\makeatletter
\renewcommand*\env@matrix[1][*\c@MaxMatrixCols c]{%
  \hskip -\arraycolsep
  \let\@ifnextchar\new@ifnextchar
  \array{#1}}
\makeatother

\title{Heritage and Irish Folklore}
\author{SETU - South East Technological University\\Imbriani Paolo - VR500437\\Professor Bridget O'Conneil}

\begin{document}

\begin{figure}
    \centering
    \includegraphics[width=0.6\textwidth]{SETU.png}
    \label{fig:centered-image}
\end{figure}

\maketitle 

\pagebreak

\tableofcontents

\pagebreak

\section{Baile, Cell \& the built enviroment}

Many place-names in Irish tell you about the history or appeareance of the location. Sometimes the root words that make 
up a place-name can be difficult to see in modern english spellings.
\\
For example "Baile" means "settlment" or "town" and is often anglicised as "Bally". "Cell" means "church" and is often anglicised as "Kil".
\textbf{Dublin} is also called \textbf{Baile Atha Cliath} which means "town of the ford of the hurdles". The name comes from the viking settlement that was established in the 9th century.
\begin{itemize}
    \item Fun fact: Baile can also mean "frenzy" or "madness" in Irish.
\end{itemize}
For example, Kilkenny is called \textbf{Cill Chainnigh} which means "church of St. Cainnech". 
Another example, Kildare is called \textbf{Cill Dara} which means "church of the oak". Saint Brigid founded a monastery in Kildare in the 5th century and was in she was a Goddess.
She is celebrated either in Catholicism and Paganism.

\subsection{Skellig Michael}

Skellig Micheal is a small island off the coast of Kerry. It is a UNESCO World Heritage Site. It was a monastic settlement from the 6th to the 12th century. The monks lived in beehive huts. The island was abandoned in the 12th century. The island was used as a location for the Star Wars movie "The Force Awakens".

\subsection{Dùn and Ràth}

You may remmeber several words for "fortress" from the previous lecture.
Dùn and Ràth are two more words for "fortress". Dùn is often anglicised as "Doon" 
and Ràth is often anglicised as "Rath". Waterford is called \textbf{Port Láirge} which means "fort of the Vikings". The Vikings established a settlement in Waterford in the 9th century.

\section{Saint Brigid}

The feast day Saint Bridget of Kildar, also called St Brigid of Ireland, Patron Saint of Ireland, is February 1st, which is traditionally the beginnning of spring in Ireland.
The ancient celtic pagan feast is one of the quarter days of the year of the Celtic Calendar which marked the mid points between solstice dates, crucial days in the earth's journey around the sun.

\subsection{Background}

\begin{itemize}
    \item As it happens, this Celtic feast that is so much part of Irish Culture, was associated with the Celtic Goddess.
    \item Many Celtic feast days were adopted by the early Christians and their associations with the Celtic Goddesses were transferred to the Christian Saints.
    \item St. Brigid is associated with the Celtic Goddess Brigid.
    \item Her hermitage there was known as KillDara, "Church of the Oak".
    \item The customs associated with St. Brigid's Day are thought to have their origins in the pagan festival of Imbolc.
    \item For example, every other day between St. Brigid Day and St. Patrick's Day is supposed to be a fair day, according to Irish Folklore.
\end{itemize}

\subsection{St. Brigid's Cross}

In some parts of Ireland, children are still sent out on St. Bridget's Eve to pull up rushes which cannot be cut with a knife.
When the rushes are brought into the house they are strewn on the floor and over the windows and doors.
The crosses are made from rushes and are hung up in the house to protect the family from harm. They are made on January 31st, the are made from fresh rushes which are plentiful in Ireland.
When you give a St. Brigid's Cross to someone, you are giving them a blessing.

\subsection{Backstory}



\begin{itemize}
    \item Brigid was bord in 5h century Ireland
    \item She embodies in herself the prechristian celtic and christian celtic spirit. 
    \item Some scholars credit Brigid with pioneering monastic life in Ireland. 
    \item She was a \textbf{woman of the land}, she associated fertility, spring and new life.
    \item Bridig's spirit of hospitality is legendary. She is said to have fed the poor and the hungry.
    \item She challenges both men and women today to create a church and a society where men and women are equally respected.
    \item Brigid spread the values of Christ. Her life continues to sing the song of the Gospel. She continues to be remembered as an extraordinary woman of faith and to occupy an important place in the hearts of Ireland and far beyond.
\end{itemize}

\section{Celtic Ireland}

\begin{itemize}
    \item Celtic period immediately preceeds the Christian period in Ireland.
    \item Celts travelled across the European continent.
    \item In Ireland since c. 500 BC 
    \item Irish language derived from a Celtic language
\end{itemize}

\subsection{The Celts}

\begin{itemize}
    \item The Celts had a reputation for being strong, vicious warriors.
    \item Central to many of the stories of Irish mythology.
    \item Living in small groups, mini kingdoms: tuatha.
    \item Rural and agricultural, no town or big settlements
    \item Druids were the priests and judges.
    \item Raiding and plundering across the sea
\end{itemize}


\begin{itemize}
    \item The Celts were a warrior society.
    \item 
\end{itemize}

Who were the Celts? 
What do you find interesting about the Celts?
How wopuld you describe the Celtics society? 
Do you think anything remains of the Celtics period in today's Ireland?
What is your opinion of the Celtic legends we have read?
What do you think the passage graves, circle tombs etc. were used for?


\end{document}