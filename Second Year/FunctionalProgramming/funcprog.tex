\documentclass[a4paper]{article}
\usepackage{import}
\usepackage{graphicx}
\usepackage{float}
\usepackage{pgfplots}
\usepackage{listings}
\usepackage{enumitem}
\usepackage{tikz}
\usetikzlibrary{decorations.pathreplacing} % for angle arc
\usetikzlibrary{angles, quotes, calc, positioning, trees} % for drawing angles
\pgfplotsset{compat=1.18,width=10cm}
\usepackage{tikz-cd}
\usepackage{booktabs}
\usepackage{cancel}
\usepackage{amsmath}
\usepackage{csquotes}
\usepackage{gensymb}
\usepackage{forest}
\usepackage{amsthm}
\usepackage{amssymb}
\usepackage{pgfplots}
\usepackage{lipsum}
\usepackage{mdframed} 
\usepackage{color}   
\usepackage{hyperref}
\newmdtheoremenv{theo}{Theorem}
\usepackage{mathtools}
\DeclarePairedDelimiter\ceil{\lceil}{\rceil}
\DeclarePairedDelimiter\floor{\lfloor}{\rfloor}

\hypersetup{
    colorlinks=true, %set true if you want colored links
    linktoc=all,     %set to all if you want both sections and subsections linked
    linkcolor=black,  %choose some color if you want links to stand out
}

% Define theorem styles
\newtheorem{theorem}{Theorem}[section]    % Theorems numbered within sections
\newtheorem{lemma}[theorem]{Lemma}        % Lemmas use the same counter as theorems
\newtheorem{corollary}[theorem]{Corollary} % Corollaries use the same counter as theorems
\newtheorem{proposition}[theorem]{Proposition} % Proposition uses the same counter
\newtheorem{property}[theorem]{Property}
\theoremstyle{definition}
\newtheorem{definition}[theorem]{Definition} % Now uses the same counter as theorems


% Remark-style theorem
\theoremstyle{remark}
\newtheorem{remark}[theorem]{Remark}

% Boxed environment for theorems
\newmdenv[
  linewidth=0.8pt,
  roundcorner=5pt,
  linecolor=black,
  backgroundcolor=white!5,
  skipabove=\baselineskip,
  skipbelow=\baselineskip,
  innerleftmargin=10pt,
  innerrightmargin=10pt,
  innertopmargin=5pt,
  innerbottommargin=5pt
]{thmbox}

% Custom proof environment (also boxed)
\renewenvironment{proof}[1][Proof]{%
  \begin{mdframed}[linewidth=0.8pt, roundcorner=5pt, linecolor=black, skipabove=\baselineskip, skipbelow=\baselineskip, innertopmargin=5pt, innerbottommargin=5pt]%
  \noindent\textbf{#1. }%
}{%
  \end{mdframed}%
}

% Redefine theorem environments to use thmbox
\let\oldtheorem\theorem
\renewenvironment{theorem}{\begin{thmbox}\begin{oldtheorem}}{\end{oldtheorem}\end{thmbox}}

\let\oldlemma\lemma
\renewenvironment{lemma}{\begin{thmbox}\begin{oldlemma}}{\end{oldlemma}\end{thmbox}}

\let\oldcorollary\corollary
\renewenvironment{corollary}{\begin{thmbox}\begin{oldcorollary}}{\end{oldcorollary}\end{thmbox}}

\let\oldproposition\proposition
\renewenvironment{proposition}{\begin{thmbox}\begin{oldproposition}}{\end{oldproposition}\end{thmbox}}

\let\oldproperty\property
  \renewenvironment{property}{\begin{oldproperty}}{\end{oldproperty}}


% Reference shortcuts
\newcommand{\thmref}[1]{Theorem~\ref{#1}}
\newcommand{\lemref}[1]{Lemma~\ref{#1}}
\newcommand{\corref}[1]{Corollary~\ref{#1}}
\newcommand{\propref}[1]{Property~\ref{#1}} 

% To customize QED symbol
\renewcommand{\qedsymbol}{$\blacksquare$}

\usetikzlibrary{decorations.pathreplacing} % for angle arc
\usetikzlibrary{angles, quotes, calc} % for drawing angles

\usepackage{color}   %May be necessary if you want to color links
\usepackage{hyperref}
\hypersetup{
    colorlinks=true, %set true if you want colored links
    linktoc=all,     %set to all if you want both sections and subsections linked
    linkcolor=black,  %choose some color if you want links to stand out
}

\usepackage{xcolor}
\usepackage[most]{tcolorbox}
% Define a custom tcolorbox environment for examples
\newtcolorbox{examplebox}[2][]{
  colback=blue!5!white,
  colframe=blue!30!black,
  title=#2,
  boxrule=0mm,
  fonttitle=\bfseries,
  width=\textwidth,
  breakable,
  #1
}

\newtcolorbox{definizione}[2] {
  colback=green!5!white,
  colframe=green!30!black,
  title=#2,
  boxrule=0mm,
  fonttitle=\bfseries,
  width=\textwidth,
  breakable,
  #1
}

\definecolor{codegreen}{rgb}{0,0.6,0}
\definecolor{codegray}{rgb}{0.5,0.5,0.5}
\definecolor{codepurple}{rgb}{0.58,0,0.82}
\definecolor{backcolour}{rgb}{0.95,0.95,0.92}

\lstdefinestyle{mystyle}{
    backgroundcolor=\color{backcolour},   
    commentstyle=\color{codegreen},
    keywordstyle=\color{magenta},
    numberstyle=\tiny\color{codegray},
    stringstyle=\color{codepurple},
    basicstyle=\ttfamily\footnotesize,
    breakatwhitespace=false,         
    breaklines=true,                 
    captionpos=b,                    
    keepspaces=true,                 
    numbers=left,                    
    numbersep=5pt,                  
    showspaces=false,                
    showstringspaces=false,
    showtabs=false,                  
    tabsize=2
}

\lstset{style=mystyle}

\makeatletter
\renewcommand*\env@matrix[1][*\c@MaxMatrixCols c]{%
  \hskip -\arraycolsep
  \let\@ifnextchar\new@ifnextchar
  \array{#1}}
\makeatother

\title{Functional Programming}
\author{SETU - South East Technological University\\Imbriani Paolo - W20114452\\Professor Mairead Meagher}

\begin{document}

\begin{figure}
    \centering
    \includegraphics[width=0.6\textwidth]{SETU.png}
    \label{fig:centered-image}
\end{figure}

\maketitle 

\pagebreak

\tableofcontents

\pagebreak

\section{Lambda Calculus, $\alpha$-equivalence, $\beta$-reduction}

Lambda Calculus is a model of computation devised in the 1930s by Alonzo Church. Functional Programming 
languages are all based on the lambda calculus. For example, Haskell, is a \textbf{pure} functional programming language, all
its features are translatable to lambda expressions. It allows a higher degree of abstractios and composability.
\\
The lambda calculus is based on the concept of a function, and it is a simple mathematical model of computation. 
We have seen the function:
\[ \lambda x . x \]
The variable $x$ here is not semantically meaningful except in its role in that single expression. Because of this, there's a form 
of equivalence between lambda terms called $\alpha$-equivalence. Two lambda terms are $\alpha$-equivalent if they differ only in the names of their bound variables.

\[\lambda x . x\]
\[\lambda apple . apple\]
\[\lambda orange . orange\]
all mean the same thing.
When we apply a function to an argument we 
\begin{itemize}
    \item substitute the input expression for all instances of bound variables within the body of the abstraction
    \item eliminate the head of the abstraction (its only purpose was to bind a variable)
\end{itemize}
This is called $\beta$-reduction. 
\begin{examplebox}{Example}
\[(\lambda x . x) 3 \rightarrow 3\]
The \(\lambda x . x\) is a function that takes an argument and returns it. When we apply it to 3, we substitute 3 for x and eliminate the head of the abstraction.
This is function is also called the \textit{identity function}.
\[\lambda x . x + 1\]
\[(\lambda x . x + 1) 3 \rightarrow 3 + 1 = 4\]
This other function takes an argument and returns it incremented by 1.
\end{examplebox}
We can also apply our identity function to another lambda astraction:
\[(\lambda x . x) (\lambda y . y) \rightarrow \lambda y . y\]
We can use another syntax here, $[x := z]$ to indicate that $z$ will  be substituted for $x$ in the expression.
Here $z$ is the function $(\lambda y . y)$.\\
We reduce this application like this:
\[(\lambda x . x)(\lambda y . y)\]
\begin{align*}
    [x := &\lambda y . y]\\
    &\lambda y . y
\end{align*}
Once more, but this time we'll add another argument:
\[(\lambda x . x)(\lambda y . y)z\]
Applications of the lambda calculus are \textit{left associative.} That is unless specific paranthesis suggest otherwise, they associate, or group, to the left. So, it can be rewritten as:
\[((\lambda x . x)(\lambda y . y))z\]
\begin{align*}
    &((\lambda x . x)(\lambda y . y))z\\
    &[x := \lambda y . y]\\
    &(\lambda y . y)z\\
    &[y := z]\\
    &z
\end{align*}    

\subsection{Variables and multiple parameters}

Variables can be:

\begin{itemize}
    \item \textbf{Free} if they are not bound by a lambda abstraction
    \item \textbf{Bound} if they are bound by a lambda abstraction 
\end{itemize}
as the $\lambda$-calculus assumes an infinite universe of free variables. They are bound to functions in an environment, then they become 
bound by usage in an abstraction.
For example, the following lambda expression:
\[\lambda x . x*y\] 
$x$ is bound by $\lambda$ over the body $x * y$, but $y$ is a free variable. When we apply this function to an argument, nothing can be done with $y$. It remains irredicible.
\begin{examplebox}{Example}
    The following lambda expression:
    \[(\lambda x . x * y)z\]
    We apply the lambda to the argument $z$:
    \begin{align*}
        (\lambda x . x * &y)z\\
        [x := &z]\\
        &zy
    \end{align*}
    The head has been applied to the argument, and the body has been reduced. Since we know nothing about $y$ and $z$, we can't reduce it further.
\end{examplebox}
\noindent
Each lambda can only bind one parameter. To bind multiple parameters, we can nest lambdas. For example, the following lambda expression:
\[\lambda x . (\lambda y . x + y)\]
This is also called a \textit{curried} function. It takes one argument, and returns another function that takes another argument. This is useful for partial application of functions.
\begin{examplebox}{Example}
    \[(\lambda x . (\lambda y . x + y)) 3\]
    We apply the outer lambda to 3:
    \begin{align*}
        (\lambda x . (\lambda y . &x + y)) 3\\
        &[x := 3]\\
        &(\lambda y . 3 + y)
    \end{align*}
    We have reduced the outer lambda to a function that takes an argument and returns it incremented by 3.
\end{examplebox}
\subsection{Lambda Calculus in Haskell} 
How do we write lambda expression in Haskell?

\begin{figure}[H]
    \centering
    \begin{tabular}{|c|c|c|c|}
        \hline
        Named function & Lambda Calculus & Haskell & Result\\
        \hline
        Identity & $\lambda x . x$ & \texttt{\textbackslash x -> x} & \texttt{id}\\
        \hline
        f x = x + 1 & $\lambda x . x + 1$ & \texttt{\textbackslash x -> x + 1} & \texttt{(+1)}\\
        \hline
        f x y = x * y & $\lambda x . \lambda y . x * y$ & \texttt{\textbackslash x -> \textbackslash y -> x * y} & \texttt{(*)}\\
        \hline
        f xs = 'c' : xs & $\lambda xs . 'c' : xs$ & \texttt{\textbackslash xs -> 'c' : xs} & \texttt{('c':)}\\
        \hline
    \end{tabular}
\end{figure}
\noindent
Lambda functions are used extensively in Haskell, notably with high order functions.

\subsection{Encoding Lambda Calculus}

\begin{itemize}
    \item Alonzo Church is credited with the invention of the lambda calculus in the 1930s.
    \item Church encoding were developed to encode data structures and operations in the lambda calculus.
    \item Church encoding are a very powerful tool for reasoning about programs.
    \item Church found out that every concept in programming languages can be represented using functions.
    \item Everything from boolean logic, conditionals, statements, numbers and even loops can be represented using functions.
\end{itemize}
But how we encode all these construcs using only functions?\\
We'll encode the behaviour of these constructs using functions. For example, we can encode the boolean values \texttt{True} and \texttt{False} as functions that take two arguments and return the first or the second argument respectively.\\
We can encode the boolean value \texttt{True} as:
\[\lambda x . \lambda y . x\]
and the boolean value \texttt{False} as:
\[\lambda x . \lambda y . y\]
To encode the logic of the NOT operator, we can define a function that takes a boolean value and returns the opposite value.\\
The NOT operator can be encoded as:
\begin{definition}
    \[NOT := \lambda b . b \text{ False True}\]
\end{definition}
\noindent
To encode the logic of the AND operator, we can define a function that takes two boolean values and returns \texttt{True} if both are \texttt{True}, otherwise it returns \texttt{False}.\\
The AND operator can be encoded as:
\begin{definition}
    \[AND := \lambda x . \lambda y . x y \text{ False}\]
\end{definition}
\noindent
To encode the logic of the OR operator, we can define a function that takes two boolean values and returns \texttt{True} if at least one of them is \texttt{True}, otherwise it returns \texttt{False}.\\
The OR operator can be encoded as:
\begin{definition}
    \[OR := \lambda x . \lambda y . x \text{ True y}\]
\end{definition}


\end{document}