\documentclass[a4paper]{article}
\usepackage{import}
\usepackage{geometry}
\usepackage{setspace}
% Set custom margins
\onehalfspacing

\geometry{a4paper, margin=1in}
\geometry{letterpaper, left=1.5in, right=1in, top=1in, bottom=1in}
\usepackage{graphicx}
\usepackage{float}
\usepackage{pgfplots}
\usepackage{listings}
\usepackage{enumitem}
\usepackage{tikz}
\usetikzlibrary{decorations.pathreplacing} % for angle arc
\usetikzlibrary{angles, quotes, calc, positioning, trees} % for drawing angles
\pgfplotsset{compat=1.18,width=10cm}
\usepackage{tikz-cd}
\usepackage{booktabs}
\usepackage{cancel}
\usepackage{amsmath}
\usepackage{csquotes}
\usepackage{gensymb}
\usepackage{forest}
\usepackage{amsthm}
\usepackage{amssymb}
\usepackage{pgfplots}
\usepackage{lipsum}
\usepackage{mdframed} 
\usepackage{color}   
\usepackage{hyperref}
\newmdtheoremenv{theo}{Theorem}
\usepackage{mathtools}
\DeclarePairedDelimiter\ceil{\lceil}{\rceil}
\DeclarePairedDelimiter\floor{\lfloor}{\rfloor}

\hypersetup{
    colorlinks=true, %set true if you want colored links
    linktoc=all,     %set to all if you want both sections and subsections linked
    linkcolor=black,  %choose some color if you want links to stand out
}

% Define theorem styles
\newtheorem{theorem}{Theorem}[section]    % Theorems numbered within sections
\newtheorem{lemma}[theorem]{Lemma}        % Lemmas use the same counter as theorems
\newtheorem{corollary}[theorem]{Corollary} % Corollaries use the same counter as theorems
\newtheorem{proposition}[theorem]{Proposition} % Proposition uses the same counter
\newtheorem{property}[theorem]{Property}
\theoremstyle{definition}
\newtheorem{definition}[theorem]{Definition} % Now uses the same counter as theorems


% Remark-style theorem
\theoremstyle{remark}
\newtheorem{remark}[theorem]{Remark}

% Boxed environment for theorems
\newmdenv[
  linewidth=0.8pt,
  roundcorner=5pt,
  linecolor=black,
  backgroundcolor=white!5,
  skipabove=\baselineskip,
  skipbelow=\baselineskip,
  innerleftmargin=10pt,
  innerrightmargin=10pt,
  innertopmargin=5pt,
  innerbottommargin=5pt
]{thmbox}

% Custom proof environment (also boxed)
\renewenvironment{proof}[1][Proof]{%
  \begin{mdframed}[linewidth=0.8pt, roundcorner=5pt, linecolor=black, skipabove=\baselineskip, skipbelow=\baselineskip, innertopmargin=5pt, innerbottommargin=5pt]%
  \noindent\textbf{#1. }%
}{%
  \end{mdframed}%
}

% Redefine theorem environments to use thmbox
\let\oldtheorem\theorem
\renewenvironment{theorem}{\begin{thmbox}\begin{oldtheorem}}{\end{oldtheorem}\end{thmbox}}

\let\oldlemma\lemma
\renewenvironment{lemma}{\begin{thmbox}\begin{oldlemma}}{\end{oldlemma}\end{thmbox}}

\let\oldcorollary\corollary
\renewenvironment{corollary}{\begin{thmbox}\begin{oldcorollary}}{\end{oldcorollary}\end{thmbox}}

\let\oldproposition\proposition
\renewenvironment{proposition}{\begin{thmbox}\begin{oldproposition}}{\end{oldproposition}\end{thmbox}}

\let\oldproperty\property
  \renewenvironment{property}{\begin{oldproperty}}{\end{oldproperty}}


% Reference shortcuts
\newcommand{\thmref}[1]{Theorem~\ref{#1}}
\newcommand{\lemref}[1]{Lemma~\ref{#1}}
\newcommand{\corref}[1]{Corollary~\ref{#1}}
\newcommand{\propref}[1]{Property~\ref{#1}} 

% To customize QED symbol
\renewcommand{\qedsymbol}{$\blacksquare$}

\usetikzlibrary{decorations.pathreplacing} % for angle arc
\usetikzlibrary{angles, quotes, calc} % for drawing angles

\usepackage{color}   %May be necessary if you want to color links
\usepackage{hyperref}
\hypersetup{
    colorlinks=true, %set true if you want colored links
    linktoc=all,     %set to all if you want both sections and subsections linked
    linkcolor=black,  %choose some color if you want links to stand out
}

\usepackage{xcolor}
\usepackage[most]{tcolorbox}
% Define a custom tcolorbox environment for examples
\newtcolorbox{examplebox}[2][]{
  colback=blue!5!white,
  colframe=blue!30!black,
  title=#2,
  boxrule=0mm,
  fonttitle=\bfseries,
  width=\textwidth,
  breakable,
  #1
}

\newtcolorbox{definizione}[2] {
  colback=green!5!white,
  colframe=green!30!black,
  title=#2,
  boxrule=0mm,
  fonttitle=\bfseries,
  width=\textwidth,
  breakable,
  #1
}

\definecolor{codegreen}{rgb}{0,0.6,0}
\definecolor{codegray}{rgb}{0.5,0.5,0.5}
\definecolor{codepurple}{rgb}{0.58,0,0.82}
\definecolor{backcolour}{rgb}{0.95,0.95,0.92}

\lstdefinestyle{mystyle}{
    backgroundcolor=\color{backcolour},   
    commentstyle=\color{codegreen},
    keywordstyle=\color{magenta},
    numberstyle=\tiny\color{codegray},
    stringstyle=\color{codepurple},
    basicstyle=\ttfamily\footnotesize,
    breakatwhitespace=false,         
    breaklines=true,                 
    captionpos=b,                    
    keepspaces=true,                 
    numbers=left,                    
    numbersep=5pt,                  
    showspaces=false,                
    showstringspaces=false,
    showtabs=false,                  
    tabsize=2
}

\lstset{style=mystyle}

\makeatletter
\renewcommand*\env@matrix[1][*\c@MaxMatrixCols c]{%
  \hskip -\arraycolsep
  \let\@ifnextchar\new@ifnextchar
  \array{#1}}
\makeatother

\title{Analisi II}
\author{Università di Verona\\Imbriani Paolo - VR500437\\Professor Zivcovich Franco}

\usepackage{titlesec}  % For customizing section titles
\usepackage{ulem}      % For underlining text

% Make section titles bigger
\titleformat{\section}[hang]{\huge\bfseries}{\thesection}{1em}{}
\titleformat{\subsection}[hang]{\Large\normalfont\itshape}{\thesubsection}{1em}{}

%================================
% THEOREM BOX
%================================

\tcbuselibrary{theorems, skins, hooks}

\newtcbtheorem[number within=section]{Theorem}{Teorema}
{%
    enhanced,
    breakable,
    colback=white,
    frame hidden,
    boxrule=0sp,
    borderline west={2pt}{0pt}{black},
    sharp corners,
    detach title,
    before upper=\tcbtitle\par\smallskip,
    coltitle=black,
    fonttitle=\bfseries\sffamily,
    description font=\mdseries,
    separator sign none,
    segmentation style={solid, black}
}
{th}

%================================
% EXAMPLE BOX
%================================

\newtcbtheorem[number within=section]{Example}{\faPencil $\;$ Esempio}
{%
    colback=blue!5!white,
    breakable,
    colframe=blue!30!black,
    coltitle=blue!40!black,
    boxrule=1pt,
    sharp corners,
    detach title,
    before upper=\tcbtitle\par\smallskip,
    fonttitle=\bfseries,
    description font=\mdseries,
    separator sign none,
    description delimiters parenthesis
}
{ex}

%================================
% DEFINITION BOX
%================================

\newtcbtheorem[number within=section]{Definition}{\faHandORight $\;$ Definizione}
{%
    enhanced,
    before skip=2mm,
    after skip=2mm,
    colback=green!5,
    colframe=green!80!black,
    boxrule=0.5mm,
    attach boxed title to top left={xshift=1cm, yshift*=1mm-\tcboxedtitleheight},
    varwidth boxed title*=-3cm,
    boxed title style={frame code={
        \path[fill=tcbcolback]
        ([yshift=-1mm,xshift=-1mm]frame.north west)
        arc[start angle=0,end angle=180,radius=1mm]
        ([yshift=-1mm,xshift=1mm]frame.north east)
        arc[start angle=180,end angle=0,radius=1mm];
        
        \path[left color=tcbcolback!60!black, right color=tcbcolback!60!black, middle color=tcbcolback!80!black]
        ([xshift=-2mm]frame.north west) -- ([xshift=2mm]frame.north east)
        [rounded corners=1mm]-- ([xshift=1mm,yshift=-1mm]frame.north east)
        -- (frame.south east) -- (frame.south west)
        -- ([xshift=-1mm,yshift=-1mm]frame.north west)
        [sharp corners]-- cycle;
    }, interior engine=empty},
    fonttitle=\bfseries,
    title={#2},
    #1
}
{def}

\newcommand{\thm}[3][]{\begin{Theorem}{#2}{#1}#3\end{Theorem}}
\newcommand{\ex}[3][]{\begin{Example}{#2}{#1}#3\end{Example}}
\newcommand{\dfn}[3][]{\begin{Definition}[colbacktitle=green!75!black]{#2}{#1}#3\end{Definition}}


\usepackage[italian]{babel}

\addto\captionsitalian{% Replace "english" with the language you use
  \renewcommand{\contentsname}%
    {Indice}%
}


\begin{document}

\begin{figure}
    \centering
    \includegraphics[width=0.3\textwidth]{../UniversityofVerona.png}
\end{figure}

\maketitle 

\pagebreak

\tableofcontents

\pagebreak

\section{Equazioni differenziali}

\subsection{Modelli differenziali}

La fisica, per descrivere dei fenomeni fisici usano la matematica e in particolare le equazioni differenziali. Infatti, si denota $x(t)$ lo spostamento nel tempo.
Con la derivata prima $x'(t)$ si denota la velocità della particella in quell'istante e con la derivata seconda $x''(t)$ l'accelerazione.
Quindi quando andiamo a tradurre
matematicamente le leggi che governano modelli naturali può essere naturale dover lavorare
con equazioni che coinvolgono una funzione incognita e qualcuna delle sue derivate.

\ex{}{La seconda legge del moto di Newton $F = ma$, che stabilisce la posizione
$x(t)$ al tempo $t$ di un corpo di massa m costante, soggetto a una forza $F(t)$, deve soddisfare
l'equazione differenziale:
\[m\frac{d^2x}{dt^2} = F(t) \; \; \text{equazione del moto}\]
}
\noindent
Quindi le equazioni differenziali nascono per descrivere fenomeni fisici e naturali. 
Possono essere classificate in modi diversi. Abbiamo infatti:
\begin{enumerate}
    \item \textit{Equazioni differenziali ordinarie} (ODE) se vengono coinvolte solo le derivate rispetto ad una sola variabile oppure \textit{equazioni differenziali parziali} (PDE) se vengono coinvolte derivate parziali
     dell'incognita rispetto a più variabili.
     \ex{}{L'equazione: \[\frac{\partial^2 u}{\partial t^2}=c^2 \frac{\partial^2u}{\partial x^2}\]
     rappresenta l'equazione delle onde che modellizza lo spostamento trasversale $u(x, t)$ nel
punto $x$ al tempo $t$ di una corda tesa che può vibrare.
     }
    \item Classificazione in base all'ordine: l'ordine di una ED è l'ordine massimo di derivazione che
    compare nell'equazione.
\ex{}{L'equazione: \[\frac{dy^2}{dt^2} + ty^3 - \cos{y} = \sin{t} \; \; \; \; \; \; \text{è di ordine 2}\]
 \[\frac{d^3y}{dt^3} - 2t\left(\frac{dy}{dt}\right)^2  = y \frac{dy^2}{dt^2} - e^t\; \; \; \; \; \; \text{è di ordine 3}\]
}
Possiamo dunque formalizzare i concetti finora introdotti attraverso la seguente definizione:
\end{enumerate}
\dfn{Equazione differenziale}{
Si dice \textbf{equazione differenziale} di ordine $n$ un'equazione del tipo
\begin{equation}
    F(t,y',y'', \dots , y^{(n)}) = 0   
\end{equation}
dove $y(t)$ è la funzione incognita e $F$ è una funzione assegnata delle $n + 2$ variabili $t, y, y'
, \dots , y(n)$ a valori reali.
\\
Si dice \textbf{ordine} di un'equazione differenziale il massimo ordine di derivazione che compare nell'equazione.\\
Si dice \textbf{soluzione (o curva integrale)} di (1) nell'intervallo $I \subset \mathbb{R}$ una funzione $\varphi$, definita almeno in $I$ e a valori reali per cui risulti:
\begin{equation*}
F(t, \varphi'(t), \varphi''(t), \dots , \varphi^{(n)}(t)) = 0 \; \; \; \; \; \; \forall t \in I
\end{equation*}
Infine si dice integrale generale dell'equazione (1) una formula che rappresenti la famiglia
di tutte le soluzioni dell'equazione (1), eventualmente al variare di uno o più parametri in essa
contenuti.
}
\ex{}{
Consideriamo una popolazione di individui, animali o vegetali che siano,
e sia $N(t)$ il numero degli individui. Osserviamo che N è funzione di del tempo $t$, assume solo
valori interi ed è a priori una funzione discontinua di t; tuttavia può essere approssimata da
una funzione continua e derivabile purché il numero degli individui sia abbastanza grande.
Supponiamo che la popolazione sia isolata e che la proporzione degli individui in età riproduttiva e la fecondità siano costanti.
Se escludiamo i casi di morte, immigrazione, emigrazione, allora il tasso di accrescimento coincide con quello di natalità 
e se indichiamo con $\lambda$ il tasso specifico di natalità (i.e. il numero
di nati per unità di tempo) l'equazione che descrive il modello diventa:
\[\frac{dN}{dt} = \lambda N(t)\]
Questo processo risulta realistico solo in popolazioni che crescono in situazioni ideali e sono
assenti tutti i fattori che ne impediscono la crescita.
}
La stessa equazione compare anche in altri modelli relativi a sistemi fisiologici ed ecologici.
\ex{}{Studiamo ora il modello di crescita (dovuto a Malthus, 1978) relativo
all'evoluzione di una popolazione isolata in presenza di risorse limitate ed in assenza di predatori
o antagonisti all'utilizzo delle risorse. In questo caso l'equazione che si ottiene è la seguente:
\begin{equation*}
    \frac{dN}{dt} = \lambda N(t) - \mu N(t)
\end{equation*}
dove come prima $\lambda$ è il tasso di natalità mentre $\mu$ è il tasso di mortalità (cioè rispettivamente
il numero di nati e morti nell'unità di tempo). Il numero $\varepsilon = \lambda - \mu$ è detto \textbf{potenziale
biologico.}
}
\noindent
Ci chiediamo ora come possiamo trovare una soluzione del problema studiato nell'Esempio
1.5. Supponiamo per il momento che sia $N \neq 0$. Allora:
\begin{equation*}
N = \varepsilon N = \frac{N}{N} = \varepsilon \Longrightarrow \frac{d}{dt} (\log{|N|}) = \varepsilon,
\end{equation*}
da cui otteniamo:
\begin{equation*}
\log{|N(t)|} = \varepsilon t + c_1 \Longrightarrow |N(t)| = e^{c_1}e^{\varepsilon t} =: k^2 e^{\varepsilon t}
\end{equation*}
dove abbiamo posto $e^{c_1} =: k^2 > 0$ costante positiva e arbitraria. A questo punto allora:
\[N(t) = \pm k^2e^{\varepsilon t}\]
\end{document}