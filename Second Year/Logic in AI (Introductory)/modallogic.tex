\documentclass[a4paper]{article}
\usepackage{import}
\usepackage{graphicx}
\usepackage{float}
\usepackage{pgfplots}
\usepackage{listings}
\usepackage{enumitem}
\usepackage{tikz}
\usetikzlibrary{decorations.pathreplacing} % for angle arc
\usetikzlibrary{angles, quotes, calc, positioning, trees} % for drawing angles
\pgfplotsset{compat=1.18,width=10cm}
\usepackage{tikz-cd}
\usepackage{booktabs}
\usepackage{cancel}
\usepackage{amsmath}
\usepackage{csquotes}
\usepackage{gensymb}
\usepackage{forest}
\usepackage{amsthm}
\usepackage{amssymb}
\usepackage{pgfplots}
\usepackage{lipsum}
\usepackage{mdframed} 
\usepackage{color}   
\usepackage{hyperref}
\newmdtheoremenv{theo}{Theorem}
\usepackage{mathtools}
\DeclarePairedDelimiter\ceil{\lceil}{\rceil}
\DeclarePairedDelimiter\floor{\lfloor}{\rfloor}

\hypersetup{
    colorlinks=true, %set true if you want colored links
    linktoc=all,     %set to all if you want both sections and subsections linked
    linkcolor=black,  %choose some color if you want links to stand out
}

% Define theorem styles
\newtheorem{theorem}{Theorem}[section]    % Theorems numbered within sections
\newtheorem{lemma}[theorem]{Lemma}        % Lemmas use the same counter as theorems
\newtheorem{corollary}[theorem]{Corollary} % Corollaries use the same counter as theorems
\newtheorem{proposition}[theorem]{Proposition} % Proposition uses the same counter
\newtheorem{property}[theorem]{Property}
\theoremstyle{definition}
\newtheorem{definition}[theorem]{Definition} % Now uses the same counter as theorems


% Remark-style theorem
\theoremstyle{remark}
\newtheorem{remark}[theorem]{Remark}

% Boxed environment for theorems
\newmdenv[
  linewidth=0.8pt,
  roundcorner=5pt,
  linecolor=black,
  backgroundcolor=white!5,
  skipabove=\baselineskip,
  skipbelow=\baselineskip,
  innerleftmargin=10pt,
  innerrightmargin=10pt,
  innertopmargin=5pt,
  innerbottommargin=5pt
]{thmbox}

% Custom proof environment (also boxed)
\renewenvironment{proof}[1][Proof]{%
  \begin{mdframed}[linewidth=0.8pt, roundcorner=5pt, linecolor=black, skipabove=\baselineskip, skipbelow=\baselineskip, innertopmargin=5pt, innerbottommargin=5pt]%
  \noindent\textbf{#1. }%
}{%
  \end{mdframed}%
}

% Redefine theorem environments to use thmbox
\let\oldtheorem\theorem
\renewenvironment{theorem}{\begin{thmbox}\begin{oldtheorem}}{\end{oldtheorem}\end{thmbox}}

\let\oldlemma\lemma
\renewenvironment{lemma}{\begin{thmbox}\begin{oldlemma}}{\end{oldlemma}\end{thmbox}}

\let\oldcorollary\corollary
\renewenvironment{corollary}{\begin{thmbox}\begin{oldcorollary}}{\end{oldcorollary}\end{thmbox}}

\let\oldproposition\proposition
\renewenvironment{proposition}{\begin{thmbox}\begin{oldproposition}}{\end{oldproposition}\end{thmbox}}

\let\oldproperty\property
  \renewenvironment{property}{\begin{oldproperty}}{\end{oldproperty}}


% Reference shortcuts
\newcommand{\thmref}[1]{Theorem~\ref{#1}}
\newcommand{\lemref}[1]{Lemma~\ref{#1}}
\newcommand{\corref}[1]{Corollary~\ref{#1}}
\newcommand{\propref}[1]{Property~\ref{#1}} 

% To customize QED symbol
\renewcommand{\qedsymbol}{$\blacksquare$}

\usetikzlibrary{decorations.pathreplacing} % for angle arc
\usetikzlibrary{angles, quotes, calc} % for drawing angles

\usepackage{color}   %May be necessary if you want to color links
\usepackage{hyperref}
\hypersetup{
    colorlinks=true, %set true if you want colored links
    linktoc=all,     %set to all if you want both sections and subsections linked
    linkcolor=black,  %choose some color if you want links to stand out
}

\usepackage{xcolor}
\usepackage[most]{tcolorbox}
% Define a custom tcolorbox environment for examples
\newtcolorbox{examplebox}[2][]{
  colback=blue!5!white,
  colframe=blue!30!black,
  title=#2,
  boxrule=0mm,
  fonttitle=\bfseries,
  width=\textwidth,
  breakable,
  #1
}

\newtcolorbox{definizione}[2] {
  colback=green!5!white,
  colframe=green!30!black,
  title=#2,
  boxrule=0mm,
  fonttitle=\bfseries,
  width=\textwidth,
  breakable,
  #1
}

\definecolor{codegreen}{rgb}{0,0.6,0}
\definecolor{codegray}{rgb}{0.5,0.5,0.5}
\definecolor{codepurple}{rgb}{0.58,0,0.82}
\definecolor{backcolour}{rgb}{0.95,0.95,0.92}

\lstdefinestyle{mystyle}{
    backgroundcolor=\color{backcolour},   
    commentstyle=\color{codegreen},
    keywordstyle=\color{magenta},
    numberstyle=\tiny\color{codegray},
    stringstyle=\color{codepurple},
    basicstyle=\ttfamily\footnotesize,
    breakatwhitespace=false,         
    breaklines=true,                 
    captionpos=b,                    
    keepspaces=true,                 
    numbers=left,                    
    numbersep=5pt,                  
    showspaces=false,                
    showstringspaces=false,
    showtabs=false,                  
    tabsize=2
}

\lstset{style=mystyle}

\makeatletter
\renewcommand*\env@matrix[1][*\c@MaxMatrixCols c]{%
  \hskip -\arraycolsep
  \let\@ifnextchar\new@ifnextchar
  \array{#1}}
\makeatother

\title{Logic in AI - Proof theory of Modal Logical Systems}
\author{Università di Verona\\Imbriani Paolo -VR500437\\Professor Margherita Zorzi}

\begin{document}

\begin{figure}
    \centering
    \includegraphics[width=0.3\textwidth]{UniversityofVerona.png}
\end{figure}

\maketitle 

\pagebreak

\tableofcontents

\pagebreak

\section{What is a modal logic?}

In Modal logic are added modalities: $\square, \diamond.$ If $A$ is a formula then $\square A$ and $\diamond A$ are formulas
There are many type of modalities and meaning we can give to this operator such as:

\begin{itemize}
    \item Temporal 
    \item Epistemic
    \item Doxastic
\end{itemize}

\subsection{Semantic Models with Kripke}

Structure: $\mathcal{M} = \left<\mathcal{W}, \mathcal{R}, \rho\right>$
\begin{itemize}
    \item $\mathcal{W}$ set of possible
    \item $\rho : \mathcal{W} \rightarrow 2^{AT}$
    \item $\mathcal{R} \subset \mathcal{W} x \mathcal{W}:$ accessability relation
\end{itemize}
There are different axioms in Normal Modal Logic such as:
\begin{itemize}
    \item Axiom K: $\square(A \rightarrow B) \rightarrow (\square A \rightarrow \square B)$ which is the kernel
    of the normal mode logic.
    \item Axiom D: $\square A \rightarrow \diamond B$
    \item Axiom T: $\square A \rightarrow A$
    \item Axiom 4: $\square A \rightarrow \square \square A$
    \item Axiom 5: $A \rightarrow \square \diamond A$
\end{itemize}
And what we can say about this axioms?

\begin{itemize}
    \item Axiom T, forces the relaiton R in the Kripke model to be \textit{reflexive}.
    \item Axiom 4, forces the relation R in the Kripke Model to be \textit{transitive}.
    \item Axiom D, forces the relation R in the Kripke model to be \textit{serial}.
    \item Axiom 5, forces the relation R in the Kripke model to be \textit{symmetric}. 
\end{itemize}

\subsection{Logics of Knowledge}

\textbf{Epistemic Modal Logic} where
\begin{itemize}
    \item $\square$ is a Knowledge
    \item $\diamond$ is a Belief
\end{itemize}

this logic is useful when we want to reason inconsistencies.

\begin{itemize}
    \item If we know something, that is true
    \item If we know something, we know that we know something
\end{itemize}

\subsection{Proof Theory, Deductive Systems in Modal Logic}

There are different deductive styles: sequent calculus, analytic tableaux, natural deduction.
And there are several approaches, including: labelled deductive systems and geometric/multidimensional deductive systems.
The desiderata is \textit{cut elimination/normalization and modularity}.

\vspace{1em}
\noindent
Another approach is \textbf{"Multidimensional"} systems, formulaes are enriches with a spatial coordinate (index/position)
that provies information within the proof.
In this way we can treat modalities in analogy as quantifiers are treated in first-orders systems.
Only modal operators can "change" the spatial position of the formulae.

\[\forall \leftrightarrow \square\]
\[\exists \leftrightarrow \diamond\]


\subsubsection{General Picture}

According to the assumption we make on the spatial coordinate $S$ of formulas $A^S$ we can "tune" the system to a specific (normal) modal logic:
\begin{itemize}
    \item $S$ is a sequence: by seetting some constraints on the rule $\diamond E$ and $\diamond I$ one obtains all the logic from $K$ to $S4$
    \item $S$ is a set, one captures the system $S4.2$.
    \item $S$ is a singleton set: one captures $S5$.
\end{itemize}

\subsection{Position and modalities}

\begin{lemma}
    A formula $A$ interacts with a formula $B$ with respect to variable $x$ if $x$ occurs in free in both $A$ and $B$.
\end{lemma}
We have the possibility to add or remove the $\square$ operator through modal interaction, "playing" with the spatial coordinate and costraint.
\subsubsection{Position}

\begin{definition}
    A position is a \textit{sequence of uninterpreted tokens}.
\end{definition}
\noindent
We have associative concatenations and a Successor.

\subsection{What adds S4.2?}

\begin{itemize}
    \item $S4.2 = S4 + \diamond \square A \rightarrow \square \diamond A (.2)$ 
\end{itemize}
This is still an enexplored country but this Logic describes Knowledge and Belief.

\subsection{Beyond S4: the Logic S5}

\begin{itemize}
    \item S5 = S4 + $\phi \rightarrow \square \diamond \phi \text{ or } K + T + D + \diamond A \rightarrow \square \diamond A$.
\end{itemize}
We obtain soundness and completeness $w.r.t$ Hilbert-style.

\subsection{Curry Howard Correspondence and the Intuitionistic}

Through BHK we can interpret proof as we can compute the proofs.


\end{document}

