\documentclass[a4paper]{article}
\usepackage{import}
\usepackage{graphicx}
\usepackage{float}
\usepackage{pgfplots}
\usepackage{listings}
\usepackage{enumitem}
\usepackage{tikz}
\usetikzlibrary{decorations.pathreplacing} % for angle arc
\usetikzlibrary{angles, quotes, calc, positioning, trees} % for drawing angles
\pgfplotsset{compat=1.18,width=10cm}
\usepackage{tikz-cd}
\usepackage{booktabs}
\usepackage{cancel}
\usepackage{amsmath}
\usepackage{csquotes}
\usepackage{gensymb}
\usepackage{forest}
\usepackage{amsthm}
\usepackage{amssymb}
\usepackage{pgfplots}
\usepackage{lipsum}
\usepackage{mdframed} 
\usepackage{color}   
\usepackage{hyperref}
\newmdtheoremenv{theo}{Theorem}
\usepackage{mathtools}
\DeclarePairedDelimiter\ceil{\lceil}{\rceil}
\DeclarePairedDelimiter\floor{\lfloor}{\rfloor}

\hypersetup{
    colorlinks=true, %set true if you want colored links
    linktoc=all,     %set to all if you want both sections and subsections linked
    linkcolor=black,  %choose some color if you want links to stand out
}

% Define theorem styles
\newtheorem{theorem}{Theorem}[section]    % Theorems numbered within sections
\newtheorem{lemma}[theorem]{Lemma}        % Lemmas use the same counter as theorems
\newtheorem{corollary}[theorem]{Corollary} % Corollaries use the same counter as theorems
\newtheorem{proposition}[theorem]{Proposition} % Proposition uses the same counter
\newtheorem{property}[theorem]{Property}
\theoremstyle{definition}
\newtheorem{definition}[theorem]{Definition} % Now uses the same counter as theorems


% Remark-style theorem
\theoremstyle{remark}
\newtheorem{remark}[theorem]{Remark}

% Boxed environment for theorems
\newmdenv[
  linewidth=0.8pt,
  roundcorner=5pt,
  linecolor=black,
  backgroundcolor=white!5,
  skipabove=\baselineskip,
  skipbelow=\baselineskip,
  innerleftmargin=10pt,
  innerrightmargin=10pt,
  innertopmargin=5pt,
  innerbottommargin=5pt
]{thmbox}

% Custom proof environment (also boxed)
\renewenvironment{proof}[1][Proof]{%
  \begin{mdframed}[linewidth=0.8pt, roundcorner=5pt, linecolor=black, skipabove=\baselineskip, skipbelow=\baselineskip, innertopmargin=5pt, innerbottommargin=5pt]%
  \noindent\textbf{#1. }%
}{%
  \end{mdframed}%
}

% Redefine theorem environments to use thmbox
\let\oldtheorem\theorem
\renewenvironment{theorem}{\begin{thmbox}\begin{oldtheorem}}{\end{oldtheorem}\end{thmbox}}

\let\oldlemma\lemma
\renewenvironment{lemma}{\begin{thmbox}\begin{oldlemma}}{\end{oldlemma}\end{thmbox}}

\let\oldcorollary\corollary
\renewenvironment{corollary}{\begin{thmbox}\begin{oldcorollary}}{\end{oldcorollary}\end{thmbox}}

\let\oldproposition\proposition
\renewenvironment{proposition}{\begin{thmbox}\begin{oldproposition}}{\end{oldproposition}\end{thmbox}}

\let\oldproperty\property
  \renewenvironment{property}{\begin{oldproperty}}{\end{oldproperty}}


% Reference shortcuts
\newcommand{\thmref}[1]{Theorem~\ref{#1}}
\newcommand{\lemref}[1]{Lemma~\ref{#1}}
\newcommand{\corref}[1]{Corollary~\ref{#1}}
\newcommand{\propref}[1]{Property~\ref{#1}} 

% To customize QED symbol
\renewcommand{\qedsymbol}{$\blacksquare$}

\usetikzlibrary{decorations.pathreplacing} % for angle arc
\usetikzlibrary{angles, quotes, calc} % for drawing angles

\usepackage{color}   %May be necessary if you want to color links
\usepackage{hyperref}
\hypersetup{
    colorlinks=true, %set true if you want colored links
    linktoc=all,     %set to all if you want both sections and subsections linked
    linkcolor=black,  %choose some color if you want links to stand out
}

\usepackage{xcolor}
\usepackage[most]{tcolorbox}
% Define a custom tcolorbox environment for examples
\newtcolorbox{examplebox}[2][]{
  colback=blue!5!white,
  colframe=blue!30!black,
  title=#2,
  boxrule=0mm,
  fonttitle=\bfseries,
  width=\textwidth,
  breakable,
  #1
}

\newtcolorbox{definizione}[2] {
  colback=green!5!white,
  colframe=green!30!black,
  title=#2,
  boxrule=0mm,
  fonttitle=\bfseries,
  width=\textwidth,
  breakable,
  #1
}

\definecolor{codegreen}{rgb}{0,0.6,0}
\definecolor{codegray}{rgb}{0.5,0.5,0.5}
\definecolor{codepurple}{rgb}{0.58,0,0.82}
\definecolor{backcolour}{rgb}{0.95,0.95,0.92}

\lstdefinestyle{mystyle}{
    backgroundcolor=\color{backcolour},   
    commentstyle=\color{codegreen},
    keywordstyle=\color{magenta},
    numberstyle=\tiny\color{codegray},
    stringstyle=\color{codepurple},
    basicstyle=\ttfamily\footnotesize,
    breakatwhitespace=false,         
    breaklines=true,                 
    captionpos=b,                    
    keepspaces=true,                 
    numbers=left,                    
    numbersep=5pt,                  
    showspaces=false,                
    showstringspaces=false,
    showtabs=false,                  
    tabsize=2
}

\lstset{style=mystyle}

\makeatletter
\renewcommand*\env@matrix[1][*\c@MaxMatrixCols c]{%
  \hskip -\arraycolsep
  \let\@ifnextchar\new@ifnextchar
  \array{#1}}
\makeatother

\title{Introduzione alla programmazione di Smart Contract per Ethereum}
\author{Università di Verona\\Imbriani Paolo - VR500437\\Professoressa Migliorini Sara}

\begin{document}

\begin{figure}
    \centering
    \includegraphics[width=0.3\textwidth]{UniversityofVerona.png}
\end{figure}

\maketitle 

\pagebreak

\tableofcontents

\pagebreak

\section{Blockchain}

\begin{definition}
    Una \textit{blockchain} è una catena di blocchi. I blocchi contengono informazioni sulle transazioni eseguite. 
\end{definition}

\begin{definition}
    Una \textit{transazione} è una struttura dati che racchiude informazioni circa il trasferimento di un certo quantitativo di token.
\end{definition}

\begin{examplebox}{Esempio}

    Supponiamo di voler creare un consorzio tra noi studenti, dove ci vogliamo scambiare appunti. 
    Di cosa abbiamo bisogno? 
    \begin{itemize}
        \item Un \textbf{registro} condiviso e immutabile in cui memorizziamo gli scambi che vengono effettuati
        \item Per ogni \textbf{transazioni} vengono memorizzate informazioni come: quando, cosa, quanto, ecc.
    \end{itemize}
    Chi lo tiene il registro? Un entità terza scelta dal gruppo quindi bisogna mettersi d'accordo a chi affidare questo incarico.
    In una rete piccola come il nostro consorzio, potrebbe anche andare bene fare un'accordo tra di noi. Ma quando il consorzio diventa molto grande diventa difficile scegliere un entità centrale
    che si occupi di tenere il registro. L'idea per le blockchain è quella di eliminare l'entità centrale e affidare il compito di tenere il registro a \textbf{tutti} i partecipanti della rete.
    Quindi ciascun membro mantiene una \textit{copia} personale del registro. 
\end{examplebox}
\noindent
\subsection{Caratteristiche delle blockchain}
\begin{itemize}
    \item Il \textbf{protocollo di consenso decentralizzata} permette di scrivere sul registro ma solo ad un costo. Se qualcuno "mente" riceve una penalità 
    che rende tale comportamento non conveniente. 
    \item \textbf{Immutabilità} significa che una volta che un blocco è stato scritto non può essere modificato.
\end{itemize}

\subsection{Bitcoin}

L'innovazione principale della blockchain è la definizione di un protocollo per la costruzione di una rete peer-to-peer in grado di raggiiungere un consenso
circa uno stato globale senza l'intervento di \textbf{un'entità centrale}.

\subsection{Ciclo di vita di una transazione bitcoin}

Prendiamo come esempio Alice e Bob che vogliono scambiarsi dei bitcoin.
\begin{enumerate}
    \item \underline{\textbf{Alice crea e firma la transazione}}. Dichiara che vuole spedire un certo quantitativo di bitcoin a Bob. Utilizza uno strumento chiamato \textit{wallet} per fare ciò, ovvero un software che consente di memorizzare le proprie chiavi private, 
    di mantenere l'elenco dei token in nostro possesso, ci permette di riceve, inviare e spendere le proprie criptovalute. 
    La funzionalità principale di un \textbf{wallet} è quella di firmare le transazioni e la creazione degli indirizzi. Un indirizzo può essere visto come un IBAN. La probabilità che vengano creati due indirizzi uguali è molto bassa ma non è zero.
    Quando il wallet di Alice viene avviato per la prima volta:
    \begin{itemize}
        \item Genera una sequenza finita di bit generati casualmente ma SICURI (una chiave privata segreta).
        \item Calcola un indirizzo attraverso un'estrazione della chiave privata (hashing).
        \item Mostra l'indirizzo come una stringa alfanumerica o QR Code.
        \item L'indirizzo non è un informazione sensibile: Alice può pubblicarlo sulla sua pagina web.
        \item La chiave privata è un segreto: Alice deve mantenerla al sicuro.
        \begin{definition}
            Un \textit{wallet gerarchico deterministico} è un tipo di wallet che genera una serie di chiavi private e pubbliche che vengono derivate secondo una struttura ad albero.
        \end{definition}
        \begin{definition}
            In bitcoin un \textit{Unspent Transaction Output (UTXO)} è un insieme di token che sono attualmente asseganti ad un indirizzo e che non sono stati ancora spesi.
            L'insieme degli UTXO è una parte fondamentale dello stato globale della blockchian di Bitcoin.
            \\
            Il wallet di Alice seleziona un insieme di UTXO tra quelli a disposizione di Alice come input delle transazione in modo da avere un quantitativo sufficiente di token.
              
        \end{definition}
    \end{itemize}  
    \item \underline{\textbf{Broadcast e validazione della transazione}}: Alice invia la transazione al network.
    \item \underline{\textbf{Mining o creazione dei blocchi}}: Decidono che transazioni includere nel blocco. 
    \begin{definition}
        Il \textit{mining} è il processo attraverso il quale i nodi creano nuovi blocchi a partire da un insieme di transazioni "pendenti". 
        La costruzione di un blocco richiede lavoro (risoluzione di un problema matematico complesso).
    \end{definition}
    Ogni nodo (miner) riceve dalla rete le nuove transazioni e le memorizza in un'area temporeanea chiamata \textit{mempool}. 
    Circa ogni 10 minuti, ogni nodo individualmente sceglie un gruppo di queste per tentare la creazione di un nuovo blocco.
    Durante la creazione di un nuobo blocco (mining) ciascun nodo include:
    \begin{enumerate}
        \item Le transazioni da lui selezionate
        \item Coinbase (transazione speciale che non ha input e ha come l'output l'indirizzo del miner e una quantità di token pari all'attuale mining reward, inizialmente 50 BTC ora 6.25)
        \item hashing del blocco precedente
        \item Bisogna aggiungere un Nonce (numero casuale tale per cui l'hash del blocco inizia con un certo numero di zeri all'inizio prefissato basato dalla difficoltà)
    \end{enumerate}
    \item \textbf{\underline{Transazione confermata}}: Il blocco viene aggiunto alla blockchain e 
    la transazione è confermata solo dopo 6 conferme (dopo aver aggiunto 6 blocchi dopo il mio). 
\end{enumerate}

\subsubsection{Fork Blockchain di Bitcoin}

\begin{enumerate}
    \item Stato iniziale: tutti i nodi hanno una visione comune della rete.
    \item Fork: due nodi espandono la blockchain in modo diverso. Si verifica uno "split" della rete.
    \item Uno split si espande ulteriormente.
    \item La rete riconverge e vince la rete più lunga.
\end{enumerate}
Ecco perché servono più conferme. Per garantire che ad un certo punto tutti i nodi della rete abbiano ricevuto tutti i blocchi della rete.

\section{Ethereum e Smart Contracts}

\begin{definition}
    Uno \textit{smart contract} è un accordo tra due o più parti, il cui rispetto può essere forzato automaticamente senza il bisogno di intermediario.
\end{definition}
\noindent
\textbf{Ethereum} è stato il primo progetto che ha implementato il concetto di Smart Contract in una blockchain.
\begin{definition}
    In Ethereum uno smart contract è \textit{programma} memorizzato all'interno della blockchain che viene eseguito automaticamente quando si verifica un evento (scheduling di una transazione) specifico.
\end{definition}
\begin{theorem}
    Gli smart contract di Ethereum sono scritti in un linguaggio \textbf{Turing completo} e compilati in bytecode eseguibile sulla \textbf{Ethereum Virtual Machine (EVM)}.
\end{theorem}
\subsection{Smart Contract}
Una volta che ho scritto e ho compilato il mio programma, lo devo pubblicare. Devo PAGARE per fare in modo che esso venga inserito nella blockchain.
Quindi anche l'esecuzione deve essere pagata.

\vspace{1em}
\noindent
\textbf{Osservazione. } Ethereum non è solamente una criptovaluta (ETH).
\begin{itemize}
    \item I nodi di una rete Ethereum costituiscono quello che viene definito un computer globale
    \item I nodi di Ethereum mettono a disposizione una EVM in grado di eseguire i programmi.
    \item Le risorse computazionali all'interno della EVM sono \textbf{limitate}. (non si tratta di un computer distribuito, ma di un computer replicato).
\end{itemize}


\subsection{Ethereum}
\begin{theorem}
    Ethereum può essere inteso come una macchina a stati ("infiniti") deterministica: ogni transazione produce un cambiamento da uno stato $s$ ad uno stato successivo $t$. 
\end{theorem}
\noindent
Una blockchain è un registro distribuito che memorizza transazioni, aggregate in blocchi. 
In Ethereum una transazione rappresenta un cambiamento all'interno di una memoria general purpose (una mappa che contiene qualsiasi tipo di coppia [chiave, valore]).

\vspace{1em}
\noindent
\textbf{Attenzione!} I cambiamenti apportati alla memoria devono essere deterministici, altrimenti non è possibile raggiungere un consenso.

\vspace{1em}
\noindent
Ethereum si comporta come un \textbf{computer globale decentralizzato}. Gli smart contract sono programmi che vengono eseguiti all'interno della EVM.

\begin{definition}
    La \textit{Ethereum Virtual Machine (EVM)} è una singleton globale: ciascun nodo mantiene una propria copia della EVM il cui stato deve essere consistente con la versione detenuta dagli altri nodi.
\end{definition}
\noindent
Ciascun nodo deve eseguire individualmente sulla propria copia locale le chiamate agli smart contract che sono attivate attraverso le transazioni.

\subsubsection{Externally Owned Account (EOA) e contratti}

In Ethereum esistono due tipi di indirizzi:
\begin{itemize}
    \item Externally Owned Account (EOA): indirizzo associato ad una chiave privata. Fondi associati e generati dai wallet. Hanno il controllo sull'accesso ai fondi.
    \item Account di contratti: sono associati agli smart contract. Non hanno una chiave privata associata e possono inviare e ricevere token.
\end{itemize}
\noindent
Una transazione può essere inizializzata solo da un EOA (mittente). Se il destinatiario della transazione è l'indirizzo di un contratto, lo scheduling
della transazione causa l'esecuzione del contratto della EVM.
Le transazioni possono contenere dei dati aggiuntivi che indicano quale funziona speciifica del contratto deve essere eseguita e quali parametri vanno passati.
Un contratto può - durante la sua esecuzione - richiamare le funzionalità di altri contratti.

\begin{definition}
    Deployare un contratto vuol dire registrare un contratto sulla blockchain e creare una transazione speicale che ha come indirizzo di destinazione 0x0 (chiamato indirizzo zero).
\end{definition}

\subsubsection{Transazioni di Ethereum}

Ogni volta che viene effettuata una transazione verso un indirizzo del contratto, questa causa l'esecuzione del metodo indicato sulla EVM.
Una transazione verso l'indirizzo di un contratto può contenere parametri \textbf{ether, data} o entrambi.
\begin{itemize}
    \item \textbf{Ether}: quantità di token da inviare al contratto
    \item \textbf{Data}: dati aggiuntivi che vengono passati al contratto
\end{itemize}
Anche qua abbiamo il numero \textbf{Noance} associato a ciascun account che conta quante transazioni sono state inviate da un indirizzo. Garantendo così che
tutte le transazioni mandate da un singolo indirizzo siano ordinate.
Questo perché dobbiamo garantire lo stato progressivo che è presente nella memoria.

\subsubsection{GAS}

Una valuta parallela di scambio, chiamata \textbf{Gas}, è il "carburante" per controllare la quantità massima di risorse che una transazione può consumare durante la sua esecuzione.
Ciascuna transazione che invoca una funzione di uno smart contract deve indicare:
\begin{itemize}
    \item \textbf{gasPrice}: prezzo che il mittente è disposto a pagare per unità di gas (maggiore è il prezzo, potenzialmente maggiore è la velocità dello scheduling della transazione)
    \item \textbf{gasLimit}: massimo numero di unità di Gas che si è disposti a pagare per completare la transazione.
\end{itemize}
\noindent
\textbf{Attenzione!} Se termina il Gas prima che il contratto abbia completato la sua esecuzione, quest'ultima viene terminata con un errore ed il Gas consumato fino a questo momento viene perso!

\subsection{Consenso}

\begin{definition}
    Il \textit{consenso} è un accordo generalizzato che deve essere raggiunto.
\end{definition}
\noindent
Esistono più definizioni:
\begin{definition}
    In [informatica] il \textit{consenso} è legato al problema più generale di sincronizzare dello stato di un sistema distribuito, in modo tale che i partecipanti ad un sistema
    distribuito possano raggiungere un accordo circa lo stato globale del sistema
\end{definition}

\subsection{Proof of Stake}

Chi possiede più token più ha il diritto di scrivere sulla blockchain.
La blockchain tiene traccia di una serie di \textit{validatori} chiunque possiede degli Ethereum può dientare un validatore attraverso una transazione attraverso una transazione speciale
con cui blocca i propri token in un deposito. 
I validatori possono perdere i propri depositi se mentono.

\subsubsection{Validatori di Ethereum}

Per partecipare alla rete Ethereum come validatore, un utente deve 32 ETH in un contratto di deposito a mettere in esecuzione 3 software sul proprio sistema.
\begin{itemize}
    \item Execution client (impacchetta le transazioni e le esegue localmente)
    \item Consensus client (Creazione dei blocchi che poi verranno trasmessi sulla rete)
    \item Validator client (verifica che i blocchi siano validi)
\end{itemize}
Verrai inserito in una lista di attesa e verrai scelto per scrivere sulla blockchain in base al numero di token che possiedi.
Quando un blocco è stato validato da un validatore, questo manda un voto (attestation) a favore del blocco ricevuto agli altri nodi della rete.

\vspace{1em}
\noindent
Nella PoS di Ethereum, il tempo è diviso in slot (12 secondi) ed in epoche (32 slots).

\subsection{Esecuzione di una transazione}

\begin{enumerate}
    \item Creazione e firma della transazinoe
    \item Verifica della validità della transazione (che abbia firmato e pagato abbastanza Eth)
    \item Aggiunta alla mempool locale
    \item Broadcast
\end{enumerate}

\subsubsection{Validazione di una transazione}

\begin{enumerate}
    \item Scelta del proponente del nuovo blocco
    \item Scelta ed esecuzione delle transazioni
    \item Costruzione del blocco: transazioni + payload
    \item Broadcast 
\end{enumerate}

\subsubsection{Validazione di un blocco}

\begin{enumerate}
    \item Ricezione di un nuovo blocco
    \item Ri-esecuzione delle transazioni
    \item Verifica di un blocco
    \item Broadcast
\end{enumerate}
alla fine, dobbiamo finalizzarla. Ci sta bisogno di un certo numero di slot che però sono molto più brevi rispetto a Bitcoin (12 secondi)
Ogni epoca andiamo a verificare cosa sia successo dei 32 slot precedenti e andiamo a vedere come hanno votato i validatori dei 32 slot che hanno composto l'epoca.
I validatori votano le coppie di checkpoint che considerano valide:
\begin{itemize}
    \item Se una coppia di checkpoint (primo blocco di ogni epoca) raccoglie il 66\% , il checkpoint viene aggiornato
    \item Il checkpoint più recente diventa justified
    \item Il checkpoint precedente diventa finalized
\end{itemize}
Per eliminare un blocco finalizzato, un attacante deve essere disposto a perdere 1/3 del totale degli Ether messi in stack dalla rete.
La finalizzazione richiede i 2/3 della maggioranza.
Chi vota in maniera opposto rispetto alla maggioranza perde il deposito dopo un certo numero di voti "sbagliati".
\\
I fork in una rete PoS sono molto più rari rispetto ai bitcoin.
\\
Benifici della PoS:
\begin{itemize}
    \item Migliore efficienza energetica
    \item Minori barriere per l'entrata nella rete (democratizzazione)
    \item Maggiore sicurezza: gli svantaggi economici per chi non si comporta correttamente nel PoS sono sensibilmente più alte rispetto al PoW. I validatori sono disincentivati a mentire essendo che hanno più token in stake.
\end{itemize}

\end{document}
