\documentclass[a4paper]{article}
\usepackage{graphicx}
\usepackage{float}
\usepackage{pgfplots}
\pgfplotsset{compat=1.18}
\usepackage{listings}
\usepackage{enumitem}
\usepackage{cancel}
\usepackage{amsmath}
\usepackage{amssymb}
\usepackage{ebproof}
\usepackage{pgfplots}
\usepackage{tikz-cd}
\usepackage{forest}
\usetikzlibrary{decorations.pathreplacing} % for angle arc
\usetikzlibrary{angles, quotes, calc} % for drawing angles
\usepackage{color}   %May be necessary if you want to color links
\usepackage{hyperref}
\hypersetup{
    colorlinks=true, %set true if you want colored links
    linktoc=all,     %set to all if you want both sections and subsections linked
    linkcolor=black,  %choose some color if you want links to stand out
}


\usepackage{mathtools}
\DeclarePairedDelimiter\ceil{\lceil}{\rceil}
\DeclarePairedDelimiter\floor{\lfloor}{\rfloor}

\usepackage{xcolor}
\usepackage{tcolorbox}

% Define a custom tcolorbox environment for examples
\newtcolorbox{examplebox}[2][]{
  colback=blue!2!white,   % Background color
  colframe=blue!30!black, % Border color
  title=#2,          % Title of the box
  boxrule=0mm,          % Thickness of the border                % Rounded corners
  fonttitle=\bfseries,    % Title font style
  width=\textwidth,  
  #1 % Width of the box (adjustable)
}

\newtcolorbox{definition} {
  colback=green!5!white,   % Background color
  colframe=green!75!black, % Border color
  title=Definizione,          % Title of the box
  boxrule=0.5mm,          % Thickness of the border                % Rounded corners
  fonttitle=\bfseries,    % Title font style
  width=\textwidth,       % Width of the box (adjustable)
}



\definecolor{codegreen}{rgb}{0,0.6,0}
\definecolor{codegray}{rgb}{0.5,0.5,0.5}
\definecolor{codepurple}{rgb}{0.58,0,0.82}
\definecolor{backcolour}{rgb}{0.95,0.95,0.92}

\lstdefinestyle{mystyle}{
    backgroundcolor=\color{backcolour},   
    commentstyle=\color{codegreen},
    keywordstyle=\color{magenta},
    numberstyle=\tiny\color{codegray},
    stringstyle=\color{codepurple},
    basicstyle=\ttfamily\footnotesize,
    breakatwhitespace=false,         
    breaklines=true,                 
    captionpos=b,                    
    keepspaces=true,                 
    numbers=left,                    
    numbersep=5pt,                  
    showspaces=false,                
    showstringspaces=false,
    showtabs=false,                  
    tabsize=2
}

\lstset{style=mystyle}

\makeatletter
\renewcommand*\env@matrix[1][*\c@MaxMatrixCols c]{%
  \hskip -\arraycolsep
  \let\@ifnextchar\new@ifnextchar
  \array{#1}}
\makeatother
\usepackage{tikz}
\usepackage{booktabs}
\title{Esercizi di Sistemi}
\author{Università di Verona\\Imbriani Paolo -VR500437\\Professor Francesco Visentin}

\begin{document}

\begin{figure}
    \centering
    \includegraphics[width=0.3\textwidth]{UniversityofVerona.png}
    \label{fig:centered-image}
\end{figure}

\maketitle 

\pagebreak

\tableofcontents

\pagebreak

\section{Esercizi su risposta libera e impulsiva}

\subsection{Esercizio 1}
\begin{examplebox}{Consegna}
Dato il seguente sistema a tempo continuo (LTI):
\[v''(t) - 5v'(t) - 6v(t) = u'(t) + 5u(t)\]
e le seguenti condizioni iniziali:
\[\begin{cases}
    v(0) = 3\\
    v'(0) = 1
\end{cases}\]
Calcolare la risposta libera (1) e la risposta forzata/impulsiva del sistema (2).
\end{examplebox}
1. Per calcolare la risposta libera iniziamo a calcolare il polinomio caratteristico:

    \[s^2 - 5s - 6 = (s + 1)(s - 6)\]
    \[\lambda_1 = -1, \lambda_2 = 6\]
    \[\mu_1 = 1, \mu_2 = 1, r = 2\]
    \[v_l(t) = c_1e^{-t} + c_2e^{6t}\]
Ora derivo $v(t)$:
\[v'(t) = -c_1e^{-t} + 6c_2e^{6t}\]
Applico le condizioni iniziali:
\[\begin{cases}
    v(0) = 3\\
    v'(0) = 1
\end{cases} \begin{cases}
    c_1e^{-t} + c_2e^{6t} = 3\\
    -c_1e^{-t} + 6c_2e^{6t} = 1
\end{cases} \Longrightarrow \begin{cases}
    c_1 + c_2 = 3\\
    -c_1 + 6c_2 = 1
\end{cases} \Longrightarrow \begin{cases}
    c_1 = \frac{17}{7}\\
    c_2 = \frac{4}{7}
\end{cases}\]
\begin{equation*}
    \tcbox[nobeforeafter]{\(v_l(t) = \frac{17}{7}e^{-t} + \frac{4}{7}e^{6t}\)}
\end{equation*}
2. Ora calcoliamo la risposta impulsiva. 
\[h(t) = d_0 \delta_0(t) + \left[\sum_{i=1}^r \sum_{l=0}^{\mu_i - 1} d_{i,l} e^{\lambda_it} \frac{t^l}{l!}\right]\delta_{-1}(t)\]
Essendo $n \neq m$ il sistema non è proprio perciò $d_0 = 0$.
\begin{align*}
    h(t) &= \left[\sum_{i=1}^r \sum_{l=0}^{\mu_i - 1} d_{i,l} e^{\lambda_it} \frac{t^l}{l!}\right]\delta_{-1}(t)\\
    &= (d_1e^{-t} + d_2e^{6t})\delta_{-1}(t)
\end{align*}
Calcoliamo ora le derivate di $h(t)$ (Questa in basso è un'equazione unica):
\[h'(t) = (-d_1e^{-t} + 6d_2e^{6t})\delta_{-1}(t) + (d_1e^{-t} + d_2e^{6t})\delta_{0}(t)\]
\[h''(t) = (d_1e^{-t} - 36d_2e^{6t})\delta_{-1}(t) + (-d_1e^{-t} + 6d_2e^{6t})\delta_{0}(t) + ...\]
\[... + (-d_1e^{-t} + 6d_2e^{6t})\delta_{0}(t) +  (d_1e^{-t} + d_2e^{6t})\delta'(t)\]
Ora poniamo $v(t) = h(t)$ e $u(t) = \delta(t)$ nell'equazione del sistema iniziale:
\[h''(t) - 5h'(t) - 6h(t) = \delta'(t) + 5\delta(t)\]
Sostituiamo ed elimino i gradini:
\[\overbrace{\cancel{(d_1e^{-t} - 36d_2e^{6t})\delta_{-1}(t)} + 2\delta_0(t)(-d_1e^{-t} + 6d_2e^{6t}) +  (d_1e^{-t} + d_2e^{6t})\delta'(t)}^{h''(t)} - ...\]
\[... - 5\overbrace{\left[\cancel{(-d_1e^{-t} + 6d_2e^{6t})\delta_{-1}(t)} + (d_1e^{-t} + d_2e^{6t})\delta_{0}(t)\right]}^{h'(t)} - 6\overbrace{\cancel{\left[(d_1e^{-t} + d_2e^{6t})\delta_{-1}(t)\right]}}^{h(t)}\]
\[= \delta'(t) + 5\delta(t)\]
Ora raccolgo le funzioni delta e le metto a sistema. Ricordiamo di imporre $t = 0$:
\begin{align*}
&\begin{cases}
    \cancel{\delta'(t)}(d_1e^{-t} + d_2e^{6t}) = \cancel{\delta'(t)}\\
    2\delta_0(t)(-d_1e^{-t} + 6d_2e^{6t}) - 5\delta_0(t) (d_1e^{-t} + d_2e^{6t}) = 5\delta(t)
\end{cases}\\
&\begin{cases}
    d_1 + d_2 = 1\\
    -2d_1\delta_0(t) + 12d_2\delta_0(t) - 5d_1\delta_0(t) - 5d_2\delta_0(t) = 5\delta(t)
\end{cases}\\
&\begin{cases}
    d_1 = 1 - d_2\\
    -7d_1\cancel{\delta_0(t)} + 7d_2\cancel{\delta_0(t)} = 5\cancel{\delta(t)}
\end{cases}\\
&\begin{cases}
    d_1 = \frac{1}{7}\\
    d_2 = \frac{6}{7}
\end{cases}
\end{align*}
Quindi la risposta forzata è:
\begin{equation*}
    \tcbox[nobeforeafter]{\(h(t) = \frac{1}{7}e^{-t}\delta_{-1}(t) + \frac{6}{7}e^{6t}\delta_{-1}(t)\)}
\end{equation*}

\subsection{Esercizio 2}

\begin{examplebox}{Consegna}
    Dato il seguente sistema a tempo continuo (LTI):
    \[2v''(t) - 3v'(t) - 2v(t) = 2u'(t) + u(t)\]
    e le seguenti condizioni iniziali:
    \[\begin{cases}
        v(0) = 4\\
        v'(0) = -2
    \end{cases}\]
    Calcolare la risposta libera (1) e la risposta forzata/impulsiva del sistema (2).
\end{examplebox}
1. Polinomio caratteristico:

    \[2s^2 - 3s - 2 = (2s + 1)(s - 2)\]
    \[\lambda_1 = -\frac{1}{2}, \lambda_2 = 2\]
    \[\mu_1 = 1, \mu_2 = 1, r = 2\]
    \[v_l(t) = c_1e^{-\frac{1}{2}t} + c_2e^{2t}\]
Ora derivo $v(t)$:
\[v'(t) = -\frac{1}{2}c_1e^{-\frac{1}{2}t} + 2c_2e^{2t}\]
Applico le condizioni iniziali:
\[\begin{cases}
    v(0) = 4\\
    v'(0) = -2
\end{cases} \begin{cases}
    c_1e^{-\frac{1}{2}t} + c_2e^{2t} = 4\\
    -\frac{1}{2}c_1e^{-\frac{1}{2}t} + 2c_2e^{2t} = -2 
\end{cases} \Longrightarrow
\begin{cases}
    c_1 + c_2 = 4\\
    -\frac{1}{2}c_1 + 2c_2 = -2
\end{cases}\] \[\begin{cases}
    c_1 = 4 - c_2\\
    -\frac{1}{2}(4 - c_2) + 2c_2 = -2
\end{cases}
\begin{cases}
    c_1 = 4 - c_2\\
    \frac{5}{2}c_2 = 0
\end{cases} \Longrightarrow \begin{cases}
    c_1 = 4\\
    c_2 = 0
\end{cases}\]
Quindi la risposta libera è:
\begin{equation*}
    \tcbox[nobeforeafter]{\(v_l(t) = 4e^{-\frac{1}{2}t}\)}
\end{equation*}
2. Ora calcoliamo la risposta impulsiva. 
\begin{align*}
    h(t) &= \overbrace{d_0}^{0} \delta_0(t) + \left[\sum_{i=1}^r \sum_{l=0}^{\mu_i - 1} d_{i,l} e^{\lambda_it} \frac{t^l}{l!}\right]\delta_{-1}(t)\\
    &= (d_1e^{-\frac{1}{2}t} + d_2e^{2t})\delta_{-1}(t)
\end{align*}
Calcoliamo le deritate bla bla bla mi sono rotto:
\[h'(t) = \left(-\frac{1}{2}d_1e^{-\frac{1}{2}t} + 2d_2e^{2t}\right)\delta_{-1}(t) + (d_1e^{-\frac{1}{2}t} + d_2e^{2t})\delta_{0}(t)\]
\begin{align*}
    h''(t) &= \left(\frac{1}{4}d_1e^{-\frac{1}{2}t} + 4d_2e^{2t}\right)\delta_{-1}(t) + \left(-\frac{1}{2}d_1e^{-\frac{1}{2}t} + 2d_2e^{2t}\right)\delta_{0}(t) + ...\\
    &+ \left(-\frac{1}{2}d_1e^{-\frac{1}{2}t} + 2d_2e^{2t}\right)\delta_{0}(t)+ (d_1e^{-\frac{1}{2}t} + d_2e^{2t})\delta'(t)
\end{align*}
Ora poniamo $v(t) = h(t)$ e $u(t) = \delta(t)$ nell'equazione del sistema iniziale:
 \[2h''(t) - 3h'(t) - 2h(t) = 2\delta'(t) + \delta(t)\]
Ora sostituiamo ed eliminiamo i gradini (Questa è un'equazione unica):
\[2\overbrace{\left[\cancel{\left(\frac{1}{4}d_1e^{-\frac{1}{2}t} + 4d_2e^{2t}\right)\delta_{-1}(t)} + 2\delta_0(t)\left(-\frac{1}{2}d_1e^{-\frac{1}{2}t} + 2d_2e^{2t}\right)+ (d_1e^{-\frac{1}{2}t} + d_2e^{2t})\delta'(t)\right]}^{h''(t)} -\]
\[- 3\overbrace{\left[\cancel{\left(-\frac{1}{2}d_1e^{-\frac{1}{2}t} + 2d_2e^{2t}\right)\delta_{-1}(t)} + (d_1e^{-\frac{1}{2}t} + d_2e^{2t})\delta_{0}(t)\right]}^{h'(t)} -  \cancel{2\overbrace{(d_1e^{-\frac{1}{2}t} + d_2e^{2t})\delta_{-1}(t)}^{h(t)}}\]
\[= 2\delta'(t) + \delta(t)\]
Mettiamo a sistema i corrispettivi delta:
\begin{align*}
&\begin{cases}
    \cancel{\delta'(t)}(d_1e^{-\frac{1}{2}t} + d_2e^{2t}) = \cancel{\delta'(t)}\\
    2\delta_0(t)\left(-\frac{1}{2}d_1e^{-\frac{1}{2}t} + 2d_2e^{2t}\right) + \delta_0(t)(d_1e^{-\frac{1}{2}t} + d_2e^{2t})= \delta(t)
\end{cases}\\
&\begin{cases}
    d_1 + d_2 = 1\\
    \cancel{-d_1\delta_0(t)} + 4d_2\delta_0(t) + \cancel{d_1\delta_0(t)} + d_2\delta_0(t) = \delta_0(t)
\end{cases}\\
&\begin{cases}
    d_1 = 1 - d_2\\
    \cancel{\delta_0(t)}5d_2 = \cancel{\delta_0(t)}
\end{cases}\\
&\begin{cases}
    d_1 = \frac{4}{5}\\
    d_2 = \frac{1}{5}
\end{cases}
\end{align*}
Quindi la risposta forzata è:
\begin{equation*}
    \tcbox[nobeforeafter]{\(h(t) = \frac{4}{5}e^{-\frac{1}{2}t}\delta_{-1}(t) + \frac{1}{5}e^{2t}\delta_{-1}(t)\)}
\end{equation*}  

\subsection{Esercizio 3}

\begin{examplebox}{Consegna}
    Dato il seguente sistema a tempo continuo (LTI):
    \[v''(t) + 2v'(t) + v(t) = u''(t) + u(t)\]
    e le seguenti condizioni iniziali:
    \[\begin{cases}
        v(0) = 4\\
        v'(0) = -2
    \end{cases}\]
    Calcolare la risposta libera (1) e la risposta forzata/impulsiva del sistema (2).
    \end{examplebox}
1. Polinomio caratteristico:
\[s^2 + 2s + 1 = (s+1)^2\]
\[\lambda_1 = -1, \mu_1 = 2\]
\[r = 1\]
\[v(t) = c_{1,0}e^{-t} + c_{1,1}e^{-t}t\]
Per semplicità chiamerò $c_{1,0} = c_1$ e $c_{1,1} = c_2$.
Ora derivo $v(t)$:
\[v'(t) = -c_1e^{-t} - c_2e^{-t}t + c_2e^{-t}\]
Applico le condizioni iniziali:
\[\begin{cases}
    v(0) = 4\\
    v'(0) = -2
\end{cases} \begin{cases}
    c_1\overbrace{e^{-t}}^{1} + \overbrace{c_2e^{-t}t}^{0} = 4\\
    -c_1\overbrace{e^{-t}}^{1} - \overbrace{c_2e^{-t}t}^{0} + c_2\overbrace{e^{-t}}^{1}= -2
\end{cases} \Longrightarrow
\begin{cases}
    c_1 = 4\\
    c_2 = 2
\end{cases}\]
Quindi la risposta libera è:
\begin{equation*}
    \tcbox[nobeforeafter]{\(v_l(t) = 4e^{-t} + 2e^{-t}t\)}
\end{equation*}
2. Ora calcoliamo la risposta impulsiva.
\begin{align*}
    h(t) &= \overbrace{d_0}^{0} \delta_0(t) + \left[\sum_{i=1}^r \sum_{l=0}^{\mu_i - 1} d_{i,l} e^{\lambda_it} \frac{t^l}{l!}\right]\delta_{-1}(t)\\
    &= (d_1e^{-t} + d_2e^{-t}t)\delta_{-1}(t)
\end{align*}
Calcoliamo le derivate di $h(t)$:
\[h'(t) = (-d_1e^{-t} + -d_2e^{-t}t + d_2e^{-t})\delta_{-1}(t) + (d_1e^{-t} + d_2e^{-t}t)\delta_{0}(t)\]
\[h''(t) = (d_1e^{-t} + d_2e^{-t}t - 2(d_2e^{-t}))\delta_{-1}t + 2\delta_0(t)((-d_1e^{-t} + -d_2e^{-t}t + d_2e^{-t}))  + ...\]
\[... + (d_1e^{-t} + d_2e^{-t}t)\delta'(t) \]
Ora poniamo $v(t) = h(t)$ e $u(t) = \delta(t)$ nell'equazione del sistema iniziale:
\[h''(t) + 2h'(t) + h(t) = \delta''(t) + \delta(t)\]
Sostituiamo ed eliminiamo i gradini:
\[\overbrace{\cancel{(d_1e^{-t} + d_2e^{-t}t - 2(d_2e^{-t}))\delta_{-1}t} + 2\delta_0(t)((-d_1e^{-t} + -d_2e^{-t}t + d_2e^{-t})) + (d_1e^{-t} + d_2e^{-t}t)\delta'(t)}^{h''(t)} + ...\]
\[2\overbrace{\left[\cancel{(-d_1e^{-t} + -d_2e^{-t}t + d_2e^{-t})\delta_{-1}(t)} + (d_1e^{-t} + d_2e^{-t}t)\delta_{0}(t) \right]}^{h'(t)} + \cancel{(d_1e^{-t} + d_2e^{-t}t)\delta_{-1}(t)}\]
\[= \delta''(t) + \delta(t)\]
Mettiamo a sistema i corrispettivi delta e poniamo $t = 0$:
\begin{align*}
&\begin{cases}
    0 = \delta''(t)\\
    (d_1e^{-t} + d_2e^{-t}t)\delta'(t) = 0\\
    2\delta_0(t)((-d_1e^{-t} + \cancel{-d_2e^{-t}t} + d_2e^{-t})) + 2\delta_0(t)(d_1e^{-t} + \cancel{d_2e^{-t}t})= \delta_0(t)
\end{cases}\\
&\begin{cases}
    d_1 = 0\\
    2\delta_0(t)(-d_1 + d_2) + 2d_1\delta_0(t) = \delta_0(t) 
\end{cases}\\
&\begin{cases}
    d_1 = 0\\
    -2d_1\cancel{\delta_0(t)} + 2d_2\cancel{\delta_0(t)} + 2d_1\cancel{\delta_0(t)} = \cancel{\delta_0(t)}
\end{cases}\\
&\begin{cases}
    d_1 = 0\\
    2d_2 = 1 \Longrightarrow d_2 = \frac{1}{2}
\end{cases}
\end{align*}
Quindi la risposta forzata è:
\begin{equation*}
    \tcbox[nobeforeafter]{\(h(t) = \frac{1}{2}e^{-t}t\delta_{-1}(t)\)}
\end{equation*}



\end{document}


