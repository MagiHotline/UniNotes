\documentclass[a4paper]{article}
\usepackage{import}
\usepackage{graphicx}
\usepackage{float}
\usepackage{pgfplots}
\usepackage{listings}
\usepackage{enumitem}
\usepackage{tikz}
\usetikzlibrary{decorations.pathreplacing} % for angle arc
\usetikzlibrary{angles, quotes, calc, positioning, trees} % for drawing angles
\pgfplotsset{compat=1.18,width=10cm}
\usepackage{tikz-cd}
\usepackage{booktabs}
\usepackage{cancel}
\usepackage{amsmath}
\usepackage{csquotes}
\usepackage{gensymb}
\usepackage{forest}
\usepackage{amsthm}
\usepackage{amssymb}
\usepackage{pgfplots}
\usepackage{lipsum}
\usepackage{mdframed} 
\usepackage{color}   
\usepackage{hyperref}
\newmdtheoremenv{theo}{Theorem}
\usepackage{mathtools}
\DeclarePairedDelimiter\ceil{\lceil}{\rceil}
\DeclarePairedDelimiter\floor{\lfloor}{\rfloor}

\hypersetup{
    colorlinks=true, %set true if you want colored links
    linktoc=all,     %set to all if you want both sections and subsections linked
    linkcolor=black,  %choose some color if you want links to stand out
}

% Define theorem styles
\newtheorem{theorem}{Theorem}[section]    % Theorems numbered within sections
\newtheorem{lemma}[theorem]{Lemma}        % Lemmas use the same counter as theorems
\newtheorem{corollary}[theorem]{Corollary} % Corollaries use the same counter as theorems
\newtheorem{proposition}[theorem]{Proposition} % Proposition uses the same counter
\newtheorem{property}[theorem]{Property}
\theoremstyle{definition}
\newtheorem{definition}[theorem]{Definition} % Now uses the same counter as theorems


% Remark-style theorem
\theoremstyle{remark}
\newtheorem{remark}[theorem]{Remark}

% Boxed environment for theorems
\newmdenv[
  linewidth=0.8pt,
  roundcorner=5pt,
  linecolor=black,
  backgroundcolor=white!5,
  skipabove=\baselineskip,
  skipbelow=\baselineskip,
  innerleftmargin=10pt,
  innerrightmargin=10pt,
  innertopmargin=5pt,
  innerbottommargin=5pt
]{thmbox}

% Custom proof environment (also boxed)
\renewenvironment{proof}[1][Proof]{%
  \begin{mdframed}[linewidth=0.8pt, roundcorner=5pt, linecolor=black, skipabove=\baselineskip, skipbelow=\baselineskip, innertopmargin=5pt, innerbottommargin=5pt]%
  \noindent\textbf{#1. }%
}{%
  \end{mdframed}%
}

% Redefine theorem environments to use thmbox
\let\oldtheorem\theorem
\renewenvironment{theorem}{\begin{thmbox}\begin{oldtheorem}}{\end{oldtheorem}\end{thmbox}}

\let\oldlemma\lemma
\renewenvironment{lemma}{\begin{thmbox}\begin{oldlemma}}{\end{oldlemma}\end{thmbox}}

\let\oldcorollary\corollary
\renewenvironment{corollary}{\begin{thmbox}\begin{oldcorollary}}{\end{oldcorollary}\end{thmbox}}

\let\oldproposition\proposition
\renewenvironment{proposition}{\begin{thmbox}\begin{oldproposition}}{\end{oldproposition}\end{thmbox}}

\let\oldproperty\property
  \renewenvironment{property}{\begin{oldproperty}}{\end{oldproperty}}


% Reference shortcuts
\newcommand{\thmref}[1]{Theorem~\ref{#1}}
\newcommand{\lemref}[1]{Lemma~\ref{#1}}
\newcommand{\corref}[1]{Corollary~\ref{#1}}
\newcommand{\propref}[1]{Property~\ref{#1}} 

% To customize QED symbol
\renewcommand{\qedsymbol}{$\blacksquare$}

\usetikzlibrary{decorations.pathreplacing} % for angle arc
\usetikzlibrary{angles, quotes, calc} % for drawing angles

\usepackage{color}   %May be necessary if you want to color links
\usepackage{hyperref}
\hypersetup{
    colorlinks=true, %set true if you want colored links
    linktoc=all,     %set to all if you want both sections and subsections linked
    linkcolor=black,  %choose some color if you want links to stand out
}

\usepackage{xcolor}
\usepackage[most]{tcolorbox}
% Define a custom tcolorbox environment for examples
\newtcolorbox{examplebox}[2][]{
  colback=blue!5!white,
  colframe=blue!30!black,
  title=#2,
  boxrule=0mm,
  fonttitle=\bfseries,
  width=\textwidth,
  breakable,
  #1
}

\newtcolorbox{definizione}[2] {
  colback=green!5!white,
  colframe=green!30!black,
  title=#2,
  boxrule=0mm,
  fonttitle=\bfseries,
  width=\textwidth,
  breakable,
  #1
}

\definecolor{codegreen}{rgb}{0,0.6,0}
\definecolor{codegray}{rgb}{0.5,0.5,0.5}
\definecolor{codepurple}{rgb}{0.58,0,0.82}
\definecolor{backcolour}{rgb}{0.95,0.95,0.92}

\lstdefinestyle{mystyle}{
    backgroundcolor=\color{backcolour},   
    commentstyle=\color{codegreen},
    keywordstyle=\color{magenta},
    numberstyle=\tiny\color{codegray},
    stringstyle=\color{codepurple},
    basicstyle=\ttfamily\footnotesize,
    breakatwhitespace=false,         
    breaklines=true,                 
    captionpos=b,                    
    keepspaces=true,                 
    numbers=left,                    
    numbersep=5pt,                  
    showspaces=false,                
    showstringspaces=false,
    showtabs=false,                  
    tabsize=2
}

\lstset{style=mystyle}

\makeatletter
\renewcommand*\env@matrix[1][*\c@MaxMatrixCols c]{%
  \hskip -\arraycolsep
  \let\@ifnextchar\new@ifnextchar
  \array{#1}}
\makeatother

\title{Physics 2 Report}
\author{SETU - South East Technological University\\Imbriani Paolo - W20114452\\Professor Joe Murphy}

\begin{document}

\begin{figure}
    \centering
    \includegraphics[width=0.6\textwidth]{SETU.png}
    \label{fig:centered-image}
\end{figure}

\maketitle 

\pagebreak

\tableofcontents

\pagebreak

\section{Lab 1 -  Specific Heat Capacity by Electical Heating}

\subsection{Introduction}

By electrically heating a lagged metal block and carefully noting the rate of heating 
and the temperature changes caused, the specific heat capacity of a metal can be evaluated
by plotting a suitable graph.
\\
Table of measurements:

\begin{center}
    
\begin{tabular}{c|c|c|c|c|c}
    \textbf{V (volt)} & \textbf{I (Ampere)} & \textbf{t (s)} & \textbf{Temp (°C)} & \textbf{$\Delta$ T} & \textbf{Q (mc$\Delta$ T)} \\
    \hline
    0 & 0 & 0 & 17 & 0 & 0\\
    8.34 & 1.34 & 120 & 20.5 & 3.5 & 2702.16\\
    8.34 & 1.34 & 240 & 27 & 10 & 5404.32\\
    8.34 & 1.38 & 360 & 35 & 18 & 8106 \\
    8.4 & 1.38 & 480 & 43 & 26 &  10808\\
    8.3 & 1.38 & 600 & 51 & 34 & 13510.8\\
    
\end{tabular}

\end{center}


\begin{figure}[H]
    \centering
    \begin{tikzpicture}
        \begin{axis}[
            axis lines=middle,
            xlabel={$t$},
            ylabel={$\Delta T$},
            xtick={2,4,6,8,10},
            ytick={1,2,4,8,16,20, 30},
            ymin=0, ymax=35,
            xmin=0, xmax=10,
            axis line style={->},
            every axis x label/.style={
                at={(ticklabel* cs:1.05)},
                anchor=west,
            },
            every axis y label/.style={
                at={(ticklabel* cs:1.05)},
                anchor=south,
            }
        ]
        
    
        % Draw the plot and points
        \addplot[thick, mark=*] coordinates {(0,0) (2, 3.5) (4, 10) (6, 18) (8, 26) (10, 34)};
        
        \end{axis}
      \end{tikzpicture}
      \caption{Graph of $\Delta T$ against t}
    \end{figure}


    \begin{figure}[H]
        \centering
        \begin{tikzpicture}
            \begin{axis}[
                axis lines=middle,
                xlabel={$\Delta T$},
                ylabel={$Q$},
                xtick={5,10,15,20,25,30,35},
                ytick={0,1000,5000,10000,15000,20000},
                ymin=0, ymax=20000,
                xmin=0, xmax=40,
                axis line style={->},
                every axis x label/.style={
                    at={(ticklabel* cs:1.05)},
                    anchor=west,
                },
                every axis y label/.style={
                    at={(ticklabel* cs:1.05)},
                    anchor=south,
                }
            ]
            
        
            % Draw the plot and points
            \addplot[thick, mark=*] coordinates {(0, 0) (3.5, 2702.16) (10, 5404.32) (18, 8106) (26, 10808) (34, 13510.8)};
            
            \end{axis}
          \end{tikzpicture}
          \caption{Graph of Q against $\Delta T$}
        \end{figure}
    \noindent
    \subsection{Conclusion}
    We found the slope coeffienct which 379,9975. We just divide the slope by the mass of the material to get the specific heat capacity of the material. The specific heat capacity of the material is 385 J/kg°C. Which exactly matches \textbf{the specific heat capacity of the material we were testing.}
    \\
\end{document}