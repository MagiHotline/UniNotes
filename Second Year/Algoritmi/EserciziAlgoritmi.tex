\documentclass[12pt]{article}
\usepackage{import}
\usepackage{graphicx}
\usepackage{float}
\usepackage{pgfplots}
\usepackage{listings}
\usepackage{enumitem}
\usepackage{tikz}
\usetikzlibrary{decorations.pathreplacing} % for angle arc
\usetikzlibrary{angles, quotes, calc, positioning, trees} % for drawing angles
\pgfplotsset{compat=1.18,width=10cm}
\usepackage{tikz-cd}
\usepackage{booktabs}
\usepackage{cancel}
\usepackage{amsmath}
\usepackage{csquotes}
\usepackage{gensymb}
\usepackage{forest}
\usepackage{amsthm}
\usepackage{amssymb}
\usepackage{pgfplots}
\usepackage{lipsum}
\usepackage{mdframed} 
\usepackage{color}   
\usepackage{hyperref}
\newmdtheoremenv{theo}{Theorem}
\usepackage{mathtools}
\DeclarePairedDelimiter\ceil{\lceil}{\rceil}
\DeclarePairedDelimiter\floor{\lfloor}{\rfloor}

\hypersetup{
    colorlinks=true, %set true if you want colored links
    linktoc=all,     %set to all if you want both sections and subsections linked
    linkcolor=black,  %choose some color if you want links to stand out
}

% Define theorem styles
\newtheorem{theorem}{Theorem}[section]    % Theorems numbered within sections
\newtheorem{lemma}[theorem]{Lemma}        % Lemmas use the same counter as theorems
\newtheorem{corollary}[theorem]{Corollary} % Corollaries use the same counter as theorems
\newtheorem{proposition}[theorem]{Proposition} % Proposition uses the same counter
\newtheorem{property}[theorem]{Property}
\theoremstyle{definition}
\newtheorem{definition}[theorem]{Definition} % Now uses the same counter as theorems


% Remark-style theorem
\theoremstyle{remark}
\newtheorem{remark}[theorem]{Remark}

% Boxed environment for theorems
\newmdenv[
  linewidth=0.8pt,
  roundcorner=5pt,
  linecolor=black,
  backgroundcolor=white!5,
  skipabove=\baselineskip,
  skipbelow=\baselineskip,
  innerleftmargin=10pt,
  innerrightmargin=10pt,
  innertopmargin=5pt,
  innerbottommargin=5pt
]{thmbox}

% Custom proof environment (also boxed)
\renewenvironment{proof}[1][Proof]{%
  \begin{mdframed}[linewidth=0.8pt, roundcorner=5pt, linecolor=black, skipabove=\baselineskip, skipbelow=\baselineskip, innertopmargin=5pt, innerbottommargin=5pt]%
  \noindent\textbf{#1. }%
}{%
  \end{mdframed}%
}

% Redefine theorem environments to use thmbox
\let\oldtheorem\theorem
\renewenvironment{theorem}{\begin{thmbox}\begin{oldtheorem}}{\end{oldtheorem}\end{thmbox}}

\let\oldlemma\lemma
\renewenvironment{lemma}{\begin{thmbox}\begin{oldlemma}}{\end{oldlemma}\end{thmbox}}

\let\oldcorollary\corollary
\renewenvironment{corollary}{\begin{thmbox}\begin{oldcorollary}}{\end{oldcorollary}\end{thmbox}}

\let\oldproposition\proposition
\renewenvironment{proposition}{\begin{thmbox}\begin{oldproposition}}{\end{oldproposition}\end{thmbox}}

\let\oldproperty\property
  \renewenvironment{property}{\begin{oldproperty}}{\end{oldproperty}}


% Reference shortcuts
\newcommand{\thmref}[1]{Theorem~\ref{#1}}
\newcommand{\lemref}[1]{Lemma~\ref{#1}}
\newcommand{\corref}[1]{Corollary~\ref{#1}}
\newcommand{\propref}[1]{Property~\ref{#1}} 

% To customize QED symbol
\renewcommand{\qedsymbol}{$\blacksquare$}

\usetikzlibrary{decorations.pathreplacing} % for angle arc
\usetikzlibrary{angles, quotes, calc} % for drawing angles

\usepackage{color}   %May be necessary if you want to color links
\usepackage{hyperref}
\hypersetup{
    colorlinks=true, %set true if you want colored links
    linktoc=all,     %set to all if you want both sections and subsections linked
    linkcolor=black,  %choose some color if you want links to stand out
}

\usepackage{xcolor}
\usepackage[most]{tcolorbox}
% Define a custom tcolorbox environment for examples
\newtcolorbox{examplebox}[2][]{
  colback=blue!5!white,
  colframe=blue!30!black,
  title=#2,
  boxrule=0mm,
  fonttitle=\bfseries,
  width=\textwidth,
  breakable,
  #1
}

\newtcolorbox{definizione}[2] {
  colback=green!5!white,
  colframe=green!30!black,
  title=#2,
  boxrule=0mm,
  fonttitle=\bfseries,
  width=\textwidth,
  breakable,
  #1
}

\definecolor{codegreen}{rgb}{0,0.6,0}
\definecolor{codegray}{rgb}{0.5,0.5,0.5}
\definecolor{codepurple}{rgb}{0.58,0,0.82}
\definecolor{backcolour}{rgb}{0.95,0.95,0.92}

\lstdefinestyle{mystyle}{
    backgroundcolor=\color{backcolour},   
    commentstyle=\color{codegreen},
    keywordstyle=\color{magenta},
    numberstyle=\tiny\color{codegray},
    stringstyle=\color{codepurple},
    basicstyle=\ttfamily\footnotesize,
    breakatwhitespace=false,         
    breaklines=true,                 
    captionpos=b,                    
    keepspaces=true,                 
    numbers=left,                    
    numbersep=5pt,                  
    showspaces=false,                
    showstringspaces=false,
    showtabs=false,                  
    tabsize=2
}

\lstset{style=mystyle}

\makeatletter
\renewcommand*\env@matrix[1][*\c@MaxMatrixCols c]{%
  \hskip -\arraycolsep
  \let\@ifnextchar\new@ifnextchar
  \array{#1}}
\makeatother

\title{Esercizi di Algoritmi}
\author{Università di Verona\\Imbriani Paolo -VR500437\\Professor Roberto Segala}

\begin{document}

\begin{figure}
    \centering
    \includegraphics[width=0.3\textwidth]{UniversityofVerona.png}
    \label{fig:centered-image}
\end{figure}

\maketitle 

\pagebreak

\tableofcontents

\pagebreak

\section{Esercizi sul libro}

\subsection{Esercizi sulla ricorrenza}

\subsubsection{Esercizio 1}
Utilizzando il metodo di sostituzione, dimostarte che la seguente ricorrenza:

\[L(n) = \begin{cases}
    1 & \text{ se } n < n_0\\
    L(n/3) + L(2n/3) & \text{ se } n \ge n_0
\end{cases}\]
Dimostrare che $L(n) \in \Omega(n)$ e deducetene che $L(n) = \Theta(n)$.
Sostituiamo: 
\[L(n) = cn\]
Per verificare che $\Omega(n)$ sia il limite inferiore.
\[L(n) = L\left(\frac{n}{3}\right) + L\left(\frac{2n}{3}\right)\] 
\[\stackrel{?}{\ge} c\frac{n}{3} + c \frac{2n}{3}\]
\[= c\left(\frac{n}{3} + \frac{2n}{3}\right)\]
\[= c\left(\frac{\cancel{3}n}{\cancel{3}}\right)\]
\[= cn\]
Siccome $cn \ge cn$ absbiamo verificato quindi che il limite asintotico inferiore è $\Omega(n)$.
Inoltre, sappiamo che $cn = cn$ quindi abbiamo verificato che $L(n) \in \Theta(n)$.

\subsection{Esercizio 2}
Utilizzando il metodo di sostituzione, dimostarte che la seguente ricorrenza:

\[T(n) = T\left(\frac{n}{3}\right) + T\left(\frac{2n}{3}\right) + \Theta(n)\]
ha soluzione $T(n) = \Omega(n \log n)$ e deducetene che $T(n) = \Theta(n \log n)$.
\[L(n) = L\left(\frac{n}{3}\right) + L\left(\frac{2n}{3}\right) + \Theta(n)\] 
\[
\begin{split}
&= \stackrel{?}{\ge} c\frac{n}{3} \log{\frac{n}{3}} + c\frac{2n}{3} \log \frac{2n}{3} + cn\\
&= c\left(\frac{n}{3}\log \frac{n}{3} + \frac{2n}{3} \log \frac{2n}{3} + n\right)\\
&= c\left(\frac{n}{3}\left(\log \frac{n}{3} + \frac{2n}{3}\log \frac{2n}{3} + 3\right)\right)\\
&= c\frac{n}{3} \left(\log \frac{n}{3} + 2\log \frac{2n}{3} + 3\right)\\
&=c\frac{n}{3} \left(\log \frac{n}{3} + 2\log\frac{n}{3} + 2\log2 + 3\right)\\
&= c\frac{n}{3}\left(3 \log \frac{n}{3} + 2\log2 +3n\right)\\
&= c\frac{n}{3}\overbrace{(3 + 2\log 2)}^{c} + cn \log \frac{n}{3}\\
&= c\frac{n}{3} + cn \log \frac{n}{3}\\
&= c\frac{n}{3} + cn \log{n}\\
&= \frac{\cancel{cn} + \cancel{cn} \log n}{\cancel{cn}} \ge \frac{\cancel{cn} \log n}{\cancel{cn}}\\
&= \log n + 1 \ge \log n\\
\end{split}
\]    



\end{document}
